\documentclass[12pt,a4paper]{report}
%\documentclass[12pt,a4paper]{book}
\usepackage[utf8]{inputenc}
\usepackage[russian]{babel}
\usepackage{amsmath}
\usepackage{amsfonts}
\usepackage{amssymb}
\usepackage{makeidx}
\usepackage{amsthm}
\usepackage{graphicx}
\usepackage{rotating}%Для поворота таблицы
\usepackage{multirow} % улучшенное форматирование таблиц
%Люблю все цветастое
\usepackage[usenames]{color}
\usepackage{colortbl}
%Люблю все цветастое
\usepackage{caption}
\usepackage{subcaption}
\usepackage{makeidx} %Для индекса 
\usepackage{wrapfig}% Для вставки картинок в текст 
\usepackage{sidecap}%Для мега капкона 
\usepackage[left=1.00cm, right=2.00cm, top=1.50cm, bottom=1.50cm]{geometry}
%%Тест цветастых картинок
\usepackage{tikz} % for \gradientbox below.
\usepackage{eso-pic}

\newcommand{\gradientbox}[3]{%
	\begin{tikzpicture}
	\node[left color=#1,right color=#2] {#3};
	\end{tikzpicture}%
}
%%
\makeindex

\begin{document}
	%Небольшая боковая панелька
	\AddToShipoutPicture*{%
		\AtPageLowerLeft{%
			\rotatebox{90}{
				\gradientbox{green!20}{white}{%
					\begin{minipage}{\paperheight}%
					\hspace*{ \stretch{1} }\textcopyright~\today \makeatletter  \quad Гребанный Вини Пух \makeatother.\hspace*{ \stretch{1} }
					\end{minipage}%
				}
			}%
		}%
	}
	%Конец небольшой боковой картинки 
\begin{titlepage}
	\centering
	\includegraphics[width=0.15\textwidth]{img/i}\par\vspace{1cm}%example-image-1x1
	{\scshape\LARGE Кафедра Винни Пуха \par}
	\vspace{1cm}
	{\scshape\Large \par}
	\vspace{1.5cm}
	{\huge\bfseries Ответы на билеты весеннего зачета \par}
	\vspace{2cm}
	{\Large\itshape Гребанный Вини Пух \par}
	\vfill
	корректор \par
	\textsc{\textbf{Рядовой \textit{Пяточек}}  }
	\vfill
	
	% Bottom of the page
	{\large \today\par}
\pagecolor{white}
\tableofcontents
\end{titlepage}

\chapter{Теоретическая часть}

\section{Основные права и свободы военнослужащего}
Военнослужащие могут состоять в общественных объединениях, не преследующих политические цели, и участвовать в их деятельности в свободное от исполнения обязанностей военной службы время.

Право на труд реализуется военнослужащими посредством прохождения военной службы.
Характер служебной деятельности и перемещения по службе военнослужащих, проходящих военную службу по призыву, определяются квалификацией и служебной необходимостью

Военнослужащие не имеют права заниматься другой оплачиваемой деятельностью, за исключением педагогической, научной и другой творческой деятельности (если она не препятствует исполнению обязанностей военной службы).

Продолжительность служебного времени военнослужащих, проходящих военную службу по призыву, определяется распорядком дня воинской части.
Им предоставляется ежедневно не менее восьми часов для сна и двух часов для личных потребностей, за исключением случаев, определённых общевойсковыми уставами (боевое дежурство, внутренняя служба и др.)

Военнослужащие имеют право на проезд на безвозмездной основе железнодорожным, воздушным, водным и автомобильным транспортом (за исключением такси) в служебные командировки, в связи с переводом на новое место военной службы, к местам использования основного отпуска, дополнительных отпусков, на лечение о обратно, на избранное место жительства при увольнении с военной службы.

Проезд на безвозмездной основе означает, что военнослужащему в воинской части выдаются воинские перевозочные документы, в обмен на которые ему в кассе соответствующего транспортного предприятия выдают бесплатный билет на тот вид транспорта, который указан в данных документах.

В случае досрочного увольнения с военной службы военнослужащего, проходящего военную службу по призыву, признанного военно-врачебной комиссией ограниченно годным к военной службе вследствие увечья (ранения, травмы, контузии) или заболевания, полученного в период прохождения военной службы, ему выплачивается 5 окладов.

Военнослужащим, прошедшим военную службу по призыву, при ух увольнении с военной службы выплачивается один оклад денежного содержания, а указанным военнослужащим из числа детей-сирот и детей, оставшихся без попечения родителей, - 5 окладов денежного содержания.

\textbf{Вывод}
\begin{enumerate}
\item Военнослужащие пользуются правами и свободами наравне с другими гражданами РФ.
\item С учётом особого характера обязанностей военнослужащих (например, неразглашение военной тайны) существуют некоторые ограничения их в общегражданских правах и свободах.
\item Во время прохождения службы по призыву и по контракту военнослужащий может получит дополнительную профессию.
\item Военнослужащий имеет право на основной отпуск и отпуск по личным обстоятельствам.
\end{enumerate}

\section{Ответсвенность военнослужащих}

Все военнослужащие независимо от воинского звания и должности равны перед законом и несут ответственность, установленную для граждан Российской Федерации, с учетом особенностей своего правового положения.

Дисциплинарную ответственность военнослужащие несут за проступки, связанные с нарушением воинской дисциплины, норм морали и воинской чести, на основании и в порядке, установленных Дисциплинарным уставом Вооруженных Сил Российской Федерации.

Административную ответственность военнослужащие несут на общих основаниях в соответствии с законодательством об административных правонарушениях. При этом к ним не могут быть применены административные взыскания в виде штрафа, исправительных работ, административного ареста и другие административные взыскания, установленные законодательством Российской Федерации.
Гражданско-правовую ответственность военнослужащие несут за неисполнение или ненадлежащее исполнение предусмотренных гражданским законодательством обязательств, за ущерб, причиненный государству, юридическим лицам, гражданам, и в других случаях, предусмотренных законодательством.

Материальную ответственность военнослужащие несут за материальный ущерб, причиненный государству при исполнении обязанностей военной службы, в соответствии с Положением о материальной ответственности военнослужащих.
Уголовную ответственность военнослужащие несут за совершенные преступления в соответствии с законодательством Российской Федерации. За преступления против установленного порядка несения военной службы они несут ответственность по закону "Об уголовной ответственности за воинские преступления".

За совершенные правонарушения военнослужащие привлекаются, как правило, к одному виду ответственности.

Военнослужащие, подвергнутые дисциплинарному взысканию в связи с совершением правонарушения, не освобождаются от уголовной ответственности за это правонарушение.
В случае совершения правонарушения, связанного с причинением материального ущерба, военнослужащие возмещают ущерб независимо от привлечения к иным видам ответственности или применения мер общественного воздействия.

Меры общественного воздействия могут быть применены к военнослужащим за проступки, связанные с нарушением ими воинской дисциплины и общественного порядка.
При привлечении к ответственности недопустимо ущемление чести и достоинства военнослужащих.
 
\section{Вооруженные силы РФ их состав и назначение}
Вооруженные Силы (ВС) РФ — государственная военная организация, предназначенная для отражения агрессии, направленной против РФ, вооруженной защиты целостности и неприкосновенности российской территории, выполнения задач в соответствии с российскими законами и международными договорами РФОткрыть в новом окне.

Решение об использовании ВС является исключительной компетенцией Президента РФ — Верховного Главнокомандующего ВС РФ. Он же осуществляет руководство, издает приказы и директивы, обязательные для исполнения всеми войсками, воинскими формированиями и органами.

\textbf{Вооруженные Силы РФ} состоят из:
\begin{itemize}
\item центральных органов военного управления;
\item объединений, соединений, воинских частей и организаций:
\item видов и родов войск ВС РФ;
\item тыла ВС РФ;
\item войск, не входящих в виды и рода войск ВС РФ.
\end{itemize}


Минобороны России является органом управления Вооруженными Силами Российской Федерации.

Президент Российской Федерации осуществляет руководство деятельностью Министерства обороныОткрыть в новом окне.

В структуру Министерства обороны входятОткрыть в новом окне:

службы Министерства обороны РФ и равные им подразделения;
центральные органы военного управления, не входящие в службы и равные им подразделения;
иные подразделения.
Минобороны России осуществляет координацию и контроль деятельности подведомственных ему федеральных органов исполнительной власти:
\begin{itemize}
\item Федеральной службы по военно-техническому сотрудничеству,
\item Федеральной службы по оборонному заказу,
\item Федеральной службы по техническому и экспортному контролю,
\item Федерального агентства специального строительства.
\end{itemize}
Министерство обороны РФ осуществляет свою деятельность непосредственно и через:
\begin{itemize}

\item органы управления военных округов;
\item иные органы военного управления;
\item территориальные органы (военные комиссариаты).
\end{itemize}
Территория РФ разбита на военные округа (главные военно-административные единицы).
 Их список можно найти в Приложении.

В состав каждого военного округа входят:
\begin{itemize}


\item органы военного управления;
\item объединения, соединения, воинские части, организации ВС;
\item военные комиссариаты (органы военного управления в городах, районах, поселках и т.п.).
\end{itemize}
Военные округа, это вопрос под номером 16, (\ref{Okruga})



\section{Обязанности солдат(матроса)}
160. Солдат (матрос) в мирное и военное время отвечает: за точное и своевременное исполнение возложенных на него обязанностей, поставленных ему задач и соблюдение при этом требований безопасности военной службы, а также за исправное состояние своего оружия, вверенной ему военной техники и сохранность выданного ему имущества. Он подчиняется командиру отделения.

161. Солдат (матрос) обязан:
глубоко сознавать свой долг воина Вооруженных Сил, образцово исполнять обязанности военной службы и соблюдать правила внутреннего порядка, овладевать всем, чему обучают командиры (начальники);

знать должности, воинские звания и фамилии своих прямых начальников до командира дивизии включительно;

оказывать уважение командирам (начальникам) и старшим, уважать честь и достоинство товарищей по службе, соблюдать правила воинской вежливости, поведения, ношения военной формы одежды и выполнения воинского приветствия;

заботиться о сохранении своего здоровья, повседневно закаливать себя, совершенствовать свою физическую подготовку, соблюдать правила личной и общественной гигиены;

в совершенстве знать и иметь всегда исправные, обслуженные, готовые к бою оружие и военную технику;

соблюдать требования безопасности военной службы на занятиях, стрельбах, учениях, при обращении с оружием и техникой, несении службы в суточном наряде и в других случаях;

знать нормативные правовые акты Российской Федерации, нормы международного гуманитарного права в пределах установленного для солдат (матросов) правового минимума, Кодекс поведения военнослужащего Вооруженных Сил — участника боевых действий, а также соответствующие международно-признанным средствам опознавания знаки различия и сигналы;

аккуратно носить обмундирование, своевременно производить его текущий ремонт, ежедневно чистить и хранить в определенном для этого месте;

при необходимости отлучиться спросить на это разрешение у командира отделения, а после возвращения доложить ему о прибытии;

при нахождении вне расположения полка вести себя с достоинством и честью, не совершать административных правонарушений, не допускать недостойных поступков по отношению к гражданскому населению.

\section{Обязанности командира отделения}

152. Командир  отделения  в мирное  и  военное  время  отвечает:
\begin{itemize}
\item за  успешное  выполнение  отделением  боевых  задач;
\item за  обучение,  воспитание,  воинскую  дисциплину  и  морально-психологическое  состояние, строевую  выправку  и  внешний  вид  подчинённых;
\item за  правильное  использование  и  сбережение  вооружения,  военной  техники, снаряжения,  обмундирования,  обуви  и за  содержание  их  в  порядке  и  исправности.
\end{itemize}
Он  подчиняется  командиру  взвода  и  его  заместителю (старшине  команды)  и  является  непосредственным  начальником  личного  состава  отделения.
 
153. Командир  отделения  обязан:
-обучать  и  воспитывать  солдат (матросов)  отделения, а  при  выполнении  боевых  задач – умело командовать  отделением;

-знать  фамилию,  имя,  отчество,  год  рождения,  национальность,  личные качества,  род  занятий  до  военной  службы,  семейное  положение, успехи  и недостатки  в  боевой  подготовке  каждого  подчинённого;

-следить  за  выполнением  распорядка  дня,  чистотой  и  внутренним  порядком  в отделении,  требовать  соблюдения  подчинёнными  воинской  дисциплины;

-знать  материальную  часть,  правила  эксплуатации  оружия,  военной  техники  и  другого  имущества  отделения,  следить  за  их  наличием,  ежедневно  осматривать  и  содержать  в  порядке  и  исправности;

-прививать  солдатам (матросам)  отделения  уважение  к службе, а  также  бережное  отношение к  своему  оружию  и  военной  технике;

-вырабатывать  у  солдат (матросов)  отделения  строевую выправку  и  физическую  выносливость;

-заботиться  о  подчинённых  и  вникать  в  их  нужды; следить  за  опрятностью,  исправностью  обмундирования  и  обуви  подчинённых. правильной  подгонкой  снаряжения,  соблюдением  ими  правил  личной  и  общественной  гигиены.  ношения  военной  формы  одежды;

-ежедневно  следить  за  чистотой  обуви,  обмундирования  и  просушкой  портянок,  носков,  а  также  за  своевременной  починкой  обуви  и  обмундирования;
следить,  чтобы  после  стрельб  и  занятий  у  подчинённых  не   оставалось  боевых  и  холостых  патронов.  гранат,  запалов  и  взрывчатых  веществ;

-докладывать заместителю  командира  взвода (старшине  команды)  о  всех  заболевших,  о  жалобах  и  просьбах  подчинённых,  об  их  проступках  и  принятых  мерах  по  их  предупреждению,  о  поощрениях  и  наложенных  на  них  дисциплинарных  взысканиях,  а  также  о  случаях  утери  или  неисправности  вооружения  и  других  материальных  средств;

-постоянно  знать,  где  находятся  и  что  делают  подчинённые.
\section{Приказ и приказания в чем различия и порядок отдачи}
39. Приказ - распоряжение командира (начальника) , обращенное к подчиненным и требующее обязательного выполнения определенных действий, соблюдения тех или иных правил или устанавливающее какой-либо порядок, положение. 
Приказ может быть отдан в письменном виде, устно или по техническим средствам связи одному или группе военнослужащих. Приказ, отданный в письменном виде, является основным распорядительным служебным документом (нормативным актом) военного управления, издаваемым на правах единоначалия командиром воинской части. Устные приказы имеют право отдавать подчиненным все командиры (начальники) . 
Обсуждение (критика) приказа недопустимо, а неисполнение приказа командира (начальника) , отданного в установленном порядке, является преступлением против военной службы. 

40. Приказание - форма доведения командиром (начальником) задач до подчиненных по частным вопросам. Приказание отдается в письменном виде или устно. Приказание, отданное в письменном виде, является распорядительным служебным документом, издаваемым начальником штаба от имени командира воинской части или военным комендантом - от имени начальника гарнизона. 

41. Приказ (приказание) должен соответствовать федеральным законам, общевоинским уставам и приказам вышестоящих командиров (начальников) . 

\section{Боевое знамя воинской части, порядок его хранения}

23. Боевое Знамя воинской части должно находиться:
\begin{enumerate}
\item при казарменном расположении воинской части и при размещении ее в населенных пунктах - в помещении штаба воинской части;

\item в воинских частях, несущих боевое дежурство (дежурство), - на командном пункте воинской части;

\item на полигоне, на учениях, в боевой обстановке - на месте, указанном командиром воинской части.

\item Разрешается совместное хранение боевых знамен нескольких воинских частей соединения.
\end{enumerate}
24. Боевое Знамя должно быть под охраной караула (дежурной смены командного пункта, пункта управления), а при выносе его к воинской части - знаменного взвода.

25. Боевое Знамя на посту (в помещении командного пункта, пункта управления) хранится в расчехленном виде на древке в застекленном шкафу, опечатанном гербовой сургучной печатью воинской части. Оно должно быть установлено в вертикальном положении в знаменную сошку (стойку с вырезами для крепления древка). При отсутствии возможности воинской части по выделению караула для охраны Боевого Знамени его разрешается хранить в опечатанном металлическом сейфе (застекленном шкафу) в секретной части.

26. При перевозке воинской части Боевое Знамя зачехляется и для него выделяется отдельное место в транспортном средстве. Вместе с Боевым Знаменем следуют знаменщик, ассистенты и караул.

27. За организацию правильного хранения и содержания Боевого Знамени непосредственно отвечает начальник штаба воинской части.

Он обязан:
\begin{enumerate}
\item систематически проверять лично или через своего заместителя и помощников несение службы часовыми на посту у Боевого Знамени;
\item производить не реже одного раза в месяц осмотр Боевого Знамени в порядке, указанном в п. 28 настоящей Временной инструкции;
\item принимать меры к устранению недостатков, обнаруженных при осмотре Боевого Знамени, докладывая об этом командиру воинской части;
\item вести специальный журнал, отмечая в нем время осмотра, недостатки, обнаруженные при осмотре Боевого Знамени, и меры, принятые к их устранению.
\end{enumerate}
28. Осмотр Боевого Знамени производится начальником штаба в присутствии знаменщика, ассистентов и начальника караула (командира дежурных сил (смен), начальника пункта управления). При этом проверяется состояние полотнища, орденов, знаменных лент, шнура с кистями, древка с навершием, скоб и подтока.

После осмотра начальник штаба опечатывает застекленный шкаф гербовой сургучной печатью и сдает Боевое Знамя под охрану караула (дежурной смены командного пункта, пункта управления).

29. Если Боевое Знамя хранится зачехленным, просушка полотнища Боевого Знамени производится вне помещения в тени или в помещении. Во время просушки Боевое Знамя охраняется знаменщиком, ассистентами и часовым (дежурной сменой командного пункта, пункта управления).

30. Учет боевых знамен ведется в воинской части, в главных штабах видов Вооруженных Сил Российской Федерации, штабе Воздушно-десантных войск, штабах военных округов, флотов и в главных и центральных управлениях Министерства обороны Российской Федерации.

\section{Военная присяга, порядок проведения к военной присяге}
  Одна из особенностей военной службы — непременное принятие военной присяги каждым гражданином, впервые зачисленным на военную службу.
  
  В наши дни поступивших на военную службу приводят к военной присяге перед Государственным флагом Российской Федерации и Боевым Знаменем воинской части. Федеральным законом Российской Федерации «О воинской обязанности и военной службе» утвержден следующий текст военной присяги:
  
  «Я, (фамилия, имя, отчество), торжественно присягаю на верность своему Отечеству — Российской Федерации.
  
  Клянусь свято соблюдать Конституцию Российской Федерации, строго выполнять требования воинских уставов, приказы командиров и начальников.
  
  Клянусь достойно исполнять воинский долг, мужественно защищать свободу, независимость и конституционный строй России, народ и Отечество».
  
  Порядок приведения к военной присяге изложен в Уставе внутренней службы Вооруженных Сил Российской Федерации. Этот ритуал проводится следующим образом.
  
  В назначенное командиром время воинская часть при Боевом Знамени и Государственном флаге Российской Федерации, с оркестром выстраивается в пешем строю в парадной, а в военное время в полевой форме одежды с оружием. Военнослужащие, приводящиеся к военной присяге, находятся в первых шеренгах. Командир воинской части в краткой речи напоминает им значение военной присяги и той почетной и ответственной обязанности, которая возлагается на них.
  
  После разъяснительной речи командир воинской части командует: «Вольно» — и приказывает командирам подразделений приступить к приведению к военной присяге.
  
  Командиры рот и других подразделений поочередно вызывают из строя военнослужащих, приводимых к военной присяге. Каждый военнослужащий читает вслух перед строем подразделения текст военной присяги, после чего собственноручно расписывается в специальном списке в графе против своей фамилии и становится на свое место в строю.
  
  По окончании церемонии командир части поздравляет воинов с приведением к военной присяге, оркестр исполняет государственный гимн. После этого воинская часть проходит торжественным маршем.
  
  Морально-нравственное и правовое значение, которое имеет для каждого военнослужащего акт принятия присяги, трудно переоценить. Военнослужащий, еще не принявший военную присягу, не может быть назначен на воинскую должность, за ним не могут быть закреплены вооружение и военная техника, он не может быть привлечен к выполнению боевых задач: участию в боевых действиях, несению боевого дежурства, боевой и караульной службы. Приняв военную присягу, военнослужащий обретает в полном объеме служебные права, но на него в полной мере возлагаются и служебные обязанности. Он принимает на себя высокую и почетную обязанность защищать свободу и независимость своей Родины и своего народа.

\section{Воинская дисциплина, чем достигается }
\begin{enumerate}

\item Воинская  дисциплина  есть  строгое  и  точное  соблюдение  всеми  военнослужащими  порядка  и правил,  установленных  законами,  воинскими  уставами  и  приказами  командиров (начальников).

\item Воинская  дисциплина  основывается  на  осознании  каждым  военнослужащим  воинского  долга  и  личной  ответственности  за  защиту  своего  Отечества,  на  его  беззаветной  преданности  своему  народу.

   Основным  методом  воспитания  у  военнослужащих  высокой  дисциплинированности  является  убеждение.  Однако  убеждение  не  исключает  применения  мер  принуждения  к  тем,  кто  недобросовестно  относится  к  выполнению  своего  воинского  долга.
\item  Воинская  дисциплина  обязывает  каждого  военнослужащего:
\begin{itemize}
\item быть  верным  Военной  присяге,  строго  соблюдать  Конституцию  и  законы  Российской  Федерации;
\item выполнять  свой  воинский  долг  умело  и   мужественно,  добросовестно  изучать  военное  дело,  беречь  военное  и  государственное  имущество;
\item стойко  переносить  трудности  военной  службы,  не  щадить  своей  жизни  для выполнения  воинского  долга;
\item быть  бдительным,  строго  хранить  военную  и  государственную  тайну;
\item  поддерживать  определённые  воинскими  уставами  правила  взаимоотношений  между  военнослужащими,  крепить  войсковое  товарищество;
\item  оказывать  уважение  командирам (начальникам)  и  друг  другу,  соблюдать  правила  воинского  приветствия  и  воинской  вежливости;
\item  с  достоинством  вести  себя  в  общественных  местах,  не  допускать  самому  и  удерживать  других от  недостойных  поступков,  содействовать  защите  чести  и  достоинства  граждан.  
\end{itemize}    
\item Высокая  воинская  дисциплина  достигается:
\begin{itemize}
\item  воспитанием  у  военнослужащих  высоких  морально – психологических  и  боевых  качеств  и  сознательного  повиновения  командирам (начальникам);
\item личной  ответственностью  каждого  военнослужащего  за  выполнение  своих  обязанностей,  и  требований  воинских  уставов;
\item  поддержанием  в  воинской  части  (подразделении)  внутреннего  порядка,  строгим  соблюдением  распорядка  дня  всеми  военнослужащими;
\item  чёткой  организацией  боевой  подготовки  и  полным  охватом  ею  личного  состава;
\item   повседневной  требовательностью  командиров (начальников)  к  подчинённым  и  контролем  за  их  исполнительностью,  уважением  личного  достоинства  военнослужащих  и  постоянной  заботой  о  них,  умелым  сочетанием  и  правильным  применением  мер  убеждения,  принуждения  и  общественного  воздействия  коллектива;
\item созданием  в  воинской  части (подразделении)  необходимых  материально – бытовых  условий.
\end{itemize}
\end{enumerate}
\section{Порядок продажи предложений, заявлений, жалоб военнослужащими.}
\textbf{106.} Военнослужащие имеют право обращаться лично, а также направлять письменные обращения (предложения, заявления или жалобы) в государственные органы, органы местного самоуправления и должностным лицам в порядке, предусмотренном законами Российской Федерации, другими нормативными правовыми актами Российской Федерации и настоящим Уставом.

Военнослужащий, которому стало известно о фактах хищения или порчи военного имущества, незаконного расходования денежных средств, злоупотреблениях, недостатках в содержании вооружения и военной техники или других фактах нанесения ущерба Вооруженным Силам Российской Федерации, обязан доложить об этом непосредственному командиру (начальнику), а также направить письменное обращение (предложение) об устранении этих недостатков или заявление (жалобу) вышестоящему командиру (начальнику).

Письменные обращения, направляемые военнослужащим должностным лицам воинской части, излагаются в форме рапорта.



\textbf{107.} Должностные лица воинской части должны внимательно относиться к поступившим обращениям (предложениям, заявлениям или жалобам). Они несут личную ответственность за своевременное их рассмотрение и принятие мер.
Должностные лица воинской части обязаны рассмотреть полученное обращение (предложение, заявление или жалобу) и, в случае если оно будет признано обоснованным, немедленно принять меры для выполнения предложения или удовлетворения просьбы подавшего обращение (предложение, заявление или жалобу), выявления и устранения вызвавших его причин, а также использовать содержащуюся в обращении (предложении, заявлении или жалобе) информацию для изучения положения дел в воинской части (подразделении).



\textbf{108. }Военнослужащий подает жалобу на незаконные в отношении его действия (бездействие) командира (начальника) или других военнослужащих, нарушение установленных законами Российской Федерации прав и свобод, неудовлетворение его положенными видами довольствия непосредственному командиру (начальнику) того лица, действия которого обжалует, а если заявляющий жалобу не знает, по чьей вине нарушены его права, жалоба подается по команде.
Военнослужащий, подавший обращение (предложение, заявление или жалобу), не освобождается от выполнения приказов и своих должностных и специальных обязанностей.



\textbf{109.} Военнослужащий, подавший обращение (предложение, заявление или жалобу), имеет право:
представлять дополнительные материалы или ходатайствовать об их истребовании командиром (начальником) или органом, рассматривающим обращение (предложение, заявление или жалобу);
знакомиться с документами и материалами, касающимися рассмотрения его обращения (предложения, заявления или жалобы), если это не затрагивает права, свободы и законные интересы других лиц и если в указанных документах и материалах не содержатся сведения, содержащие государственную или иную охраняемую федеральным законом тайну;
получать письменный ответ по существу поставленных в обращении (предложении, заявлении или жалобе) вопросов или уведомление о переадресации письменного обращения (предложения, заявления или жалобы) в иные органы или должностному лицу, в компетенцию которых входит решение указанных вопросов;
обращаться с жалобой на принятое по обращению (предложению, заявлению или жалобе) решение или на действия (бездействие) в связи с рассмотрением обращения (предложения, заявления или жалобы) в административном и (или) судебном порядке в соответствии с законодательством Российской Федерации;
обращаться с заявлением о прекращении рассмотрения обращения (предложения, заявления или жалобы).


\textbf{110.}Запрещается подавать обращение (предложение, заявление или жалобу) во время несения боевого дежурства (боевой службы), при нахождении в строю (за исключением обращений (предложений, заявлений или жалоб), подаваемых на опросе военнослужащих), в карауле, на вахте, а также в другом наряде и на занятиях.



\textbf{111.} Запрещается препятствовать подаче обращения (предложения, заявления или жалобы) военнослужащим и подвергать его за это наказанию, преследованию либо ущемлению по службе. Виновный в этом командир (начальник), так же как и военнослужащий, подавший заведомо ложное заявление (жалобу), привлекается к ответственности в соответствии с законодательством Российской Федерации.



\textbf{112.} На опросе военнослужащих обращение (предложение, заявление или жалоба) может быть заявлено устно или подано в письменном виде непосредственно лицу, проводящему опрос.
Военнослужащие, по какой-либо причине отсутствовавшие на опросе, могут подавать обращение (предложение, заявление или жалобу) в письменном виде непосредственно на имя командира (начальника), проводившего опрос.


\textbf{113.} Личный прием военнослужащих в воинских частях проводится командиром воинской части и его заместителями.
Информация о месте приема, а также установленных для приема днях и часах доводится до сведения военнослужащих в установленном порядке.
При личном приеме военнослужащий предъявляет документ, удостоверяющий его личность.
В случае если в обращении (предложении, заявлении или жалобе) содержатся вопросы, решение которых не входит в компетенцию должностного лица воинской части, военнослужащему дается разъяснение, куда и в каком порядке ему следует обратиться.
В ходе личного приема военнослужащему может быть отказано в дальнейшем рассмотрении обращения (предложения, заявления или жалобы), если ранее ему был дан ответ по существу поставленных в нем вопросов.



\textbf{114}. Если в обращении (предложении, заявлении или жалобе) содержатся вопросы, не относящиеся к компетенции должностного лица воинской части, то должностное лицо, получившее обращение (предложение, заявление или жалобу), не позднее чем в семидневный срок со дня регистрации направляет его в соответствующий орган или соответствующему должностному лицу, в компетенцию которых входит разрешение поставленных вопросов, и уведомляет об этом военнослужащего, направившего обращение (предложение, заявление или жалобу).
Запрещается пересылать обращение (предложение, заявление или жалобу) на рассмотрение тех органов или должностных лиц, действия которых обжалуются. В таких случаях обращение возвращается военнослужащему с разъяснением ему прав на обжалование соответствующих решений или действий (бездействия) в суд в установленном порядке.


\textbf{115.} Обращение (предложение, заявление или жалоба) считается разрешенным, если рассмотрены все поставленные в нем вопросы, по нему приняты необходимые меры и даны исчерпывающие ответы в соответствии с законодательством Российской Федерации.
Отказ в удовлетворении запросов, изложенных в обращении (предложении, заявлении или жалобе), доводится до сведения подавшего его военнослужащего со ссылкой на законы Российской Федерации, другие нормативные правовые акты Российской Федерации и (или) общевоинские уставы, с указанием мотивов отказа и разъяснением порядка обжалования принятого решения.



\textbf{116.} Все обращения (предложения, заявления или жалобы) подлежат обязательному рассмотрению в срок до 30 суток со дня регистрации.
В исключительных случаях, а также когда для разрешения обращения (предложения, заявления или жалобы) необходимо проведение специальной проверки, истребование дополнительных материалов или принятие других мер, срок разрешения обращения (предложения, заявления или жалобы) может быть продлен командиром воинской части, но не более чем на 30 суток, с уведомлением об этом военнослужащего, подавшего обращение (предложение, заявление или жалобу).



\textbf{117.} При рассмотрении обращения (предложения, заявления или жалобы) не допускается разглашение содержащихся в нем сведений, а также сведений, касающихся частной жизни военнослужащего, без его согласия. Не является разглашением сведений, содержащихся в обращении (предложении, заявлении или жалобе), направление этого обращения (предложения, заявления или жалобы) в орган или должностному лицу, в компетенцию которых входит решение поставленных в нем вопросов.



\textbf{118.} Командиры воинских частей обязаны не реже одного раза в квартал проводить внутреннюю проверку состояния работы по рассмотрению обращений (предложений, заявлений или жалоб). Для проведения такой проверки приказом соответствующего командира (начальника) создается комиссия. По результатам работы комиссии составляется аналитическая справка, которая хранится совместно с материалами по организации работы с обращениями (предложениями, заявлениями или жалобами) в делах воинской части.



\textbf{119.} Обращения (предложения, заявления или жалобы), поступившие в воинскую часть, в срок не более трех суток регистрируются в Книге учета письменных обращений (предложений, заявлений или жалоб) воинской части (приложение N 4) и в обязательном порядке докладываются командиру воинской части и (или) соответствующему должностному лицу.
При личном приеме содержание устного обращения (предложения, заявления или жалобы) заносится в карточку личного приема (приложение N 5), а письменное обращение (предложение, заявление или жалоба) регистрируется в установленном порядке.
Книга учета письменных обращений (предложений, заявлений или жалоб) и карточки личного приема ведутся и хранятся в штабе воинской части (органе военного управления).



\textbf{120.} Книга учета письменных обращений (предложений, заявлений или жалоб) и карточки личного приема представляются для проверки своевременности и правильности выполнения принятых решений: командиру воинской части - ежемесячно, инспектирующим (проверяющим) - по их требованию.
Книга учета письменных обращений (предложений, заявлений или жалоб) должна быть пронумерована, прошнурована, скреплена мастичной печатью и заверена командиром воинской части.

\section{Виды поощрения, налагаемые на солдат и матросов, сержантов и старшин.}
\textbf{а)} снятие  ранее  наложенного  дисциплинарного  взыскания;

\textbf{б)} объявление  благодарности;

\textbf{в)} сообщение  на  родину  или  по  месту  прежней  работы  (учёбы)  военнослужащего,  проходящего  военную  службу  по  призыву,  об  образцовом  выполнении  им  воинского  долга  и  о  полученных  поощрениях;

\textbf{г)} награждение  грамотами,  ценными  подарками  или  деньгами;

\textbf{д)} награждение  личной  фотографией военнослужащего,  снятого  при  развёрнутом  Боевом  Знамени  воинской  части (Военно – морском  флаге);

\textbf{е)} присвоение  солдатам (матросам)  воинского  звания  ефрейтор  (старший  матрос);

\textbf{ж)} присвоение  сержантам  (старшинам)  очередного  воинского  звания  на  одну  ступень  выше  воинского  звания,  предусмотренного  по  занимаемой  штатной  должности;

\textbf{з)} награждение  нагрудным  знаком  отличника;

\textbf{и)} занесение  в  Книгу  почёта  воинской  части (корабля) фамилий  солдат,  матросов,  сержантов  и  старшин;

\textbf{к)} увеличение  продолжительности  основного  отпуска  военнослужащим,  проходящим  военную  службу  по  призыву  на  срок  до  5  суток.

  К  военнослужащим,  проходящим  военную  службу  по  контракту  на  должностях  солдат,  матросов,  сержантов  и  старшин,  применяются  все  поощрения,  указанные  в  данной  статье,  кроме  пп.  \textbf{«в», «к»}.

\section{Дисциплинарные взыскания, налагаемые на солдат, матросов, сержантов и старшин}
51.\textbf{ На солдат и матросов }могут налагаться следующие взыскания:

а) выговор;\\
б) строгий выговор;
в) лишение солдат и матросов, подходящих военную службу по призыву, очередного увольнения из расположения воинской части или с корабля на берег;\\
г) назначение солдат и матросов, проходящих военную службу по призыву, вне очереди в наряд на работу - до 5 нарядов;\\
д) (изм. Указом Президента №671 от 30.06.2002г.) арест с содержанием на гауптвахте солдат и матросов, проходящих военную службу по контракту, - до 7 суток, а солдат и матросов, проходящих военную службу по призыву, - до 10 суток;\\
е) лишение нагрудного знака отличника;\\
ж) досрочное увольнение в запас солдат и матросов, проходящих военную службу по контракту.\\

52. \textbf{На сержантов и старшин}, проходящих военную службу по призыву, могут налагаться следующие взыскания:
а) выговор;\\
б) строгий выговор;\\
в) лишение очередного увольнения из расположения воинской части или с корабля на берег;\\
г) (изм. Указом Президента №671 от 30.06.2002г.) арест с содержанием на гауптвахте - до 10 суток;\\
д) лишение нагрудного знака отличника;\\
е) снижение в должности;\\
ж) снижение в воинском звании на одну степень;\\
з) снижение в воинском звании на одну степень с переводом на низшую должность.\\

53. \textbf{На сержантов и старшин}, проходящих военную службу по контракту, могут налагаться следующие взыскания.\\
а) выговор;\\
б) строгий выговор;\\
в) (изм. Указом Президента №671 от 30.06.2002г.) арест с содержанием на гауптвахте - до 7 суток;\\
г) лишение нагрудного знака отличника;\\
д) снижение в должности;\\
е) досрочное увольнение в запас.\\

На военнослужащих-женщин, проходящих военную службу в качестве солдат, матросов, сержантов и старшин, взыскания, указанные в п. "с" данной статьи и статьи 51, пп. "в" "д" не налагаются.

\section{Внутренний порядок  и размещение военнослужащих}
157. Внутренний порядок – это строгое соблюдение определённых воинскими уставами правил размещения, повседневной деятельности, быта военнослужащих в воинской части (подразделении) и несения службы суточным нарядом.\\
Внутренний порядок достигается:
\textbf{---} глубоким пониманием, сознательным и точным выполнением всеми военнослужащими обязанностей, определённых законами и воинскими уставами;\\
\textbf{---} целенаправленной воспитательной работой, сочетанием высокой требовательности командиров (начальников) с постоянной заботой о подчинённых и сохранении их здоровья;\\
\textbf{---} чёткой организацией боевой подготовки;\\
\textbf{--- }образцовым несением боевого дежурства и службы суточным нарядом;\\
\textbf{---} точным выполнением распорядка дня и регламента служебного времени;\\
\textbf{---} выполнением правил эксплуатации (использования) вооружения, военной техники и других материальных средств; созданием в местах расположения военнослужащих условий для их повседневной деятельности, жизни и быта, отвечающих требованиям воинских уставов;\\
\textbf{---} соблюдением требований пожарной безопасности, а также принятием мер по охране окружающей среды в районе деятельности воинской части.\\

\section{\textcolor{red}{!!!Порядок хранения и выдачи оружия(Надо подумать, слишком много текста выходит)}}
Надо подумать, слишком много текста

\section{Распорядок дня в воинской части}

\begin{tabular} {l | l | l | l | l}
	№ & \textbf{Мероприятия} & \textbf{Начало}  & \textbf{Конец} & \textbf{Продолжительность}  \\
	1. &  Подъем зам. ком.  взводов, ком. отделений &  \textbf{05.50} &  \textbf{06.00} & \textbf{10} \\
	2. &  Общий подъем подразделения  &  \textbf{06.00} &  \textbf{06.10} &  \textbf{10} \\
	3. &  Утренняя физическая зарядка & \textbf{06.10} & \textbf{06.40} & \textbf{30\textbf{}} \\
	4. &  Утренний туалет, заправка постелей &  \textbf{06.40} &  \textbf{07.10}  & \textbf{30} \\
	5. &  Утренний осмотр &  \textbf{07.10} &  \textbf{07.30} &  \textbf{20} \\
	6. &  З а в т р а к &  \textbf{07.30}  & \textbf{07.50} &  \textbf{20} \\
	7. &  Подготовка к занятиям  & 07.50 &  08.00 &  \textbf{10} \\
	8. &  Прослушивание радиопередач &  08.00 &  08.15  & \textbf{15} \\
	9. &  Информирование личного состава, тренировка &  08.15  & 08.45  & \textbf{30} \\
	10. & Развод личного состава, следование на занятия  & 08.45 & 09.00 & \textbf{15} \\
	11. & 1-й час занятий. ЕТО.  & 09.00 & 09.50 & \textbf{50} \\
	12. & 2-й час занятий & 10.00 & 10.50 & \textbf{50} \\
	13. & 3-й час занятий & 11.00 & 11.50 & \textbf{50} \\
	14. & 4-й час занятий & 12.00 & 12.50 & \textbf{50} \\
	15. & 5-й час занятий & 13.00 & 13.50 & \textbf{50} \\
	16. & Чистка обуви, мытье рук & 13.50 & 14.00 & \textbf{10} \\
	17. & О  б  е  д & 14.00 & 14.30 & 30 \\
	18. & Время для личных потребностей & 14.30 & 15.00 & 30 \\
	19. & 6-й час занятий, самоподготовка (подготовка к занятиям) & 15.00 & 15.50 & 50 \\
	20. & Обслуживание ВВТ & 16.00 & 17.00 & 60 \\
	21. & Смена спецодежды, чистка обуви и мытье рук & 17.00 & 17.25 & 25 \\
	22. & Подведение итогов и постановка задач & 17.25 & 17.50 & 25 \\
	23. & Воспитательные мероприятия и спортивно-массовая работа & 18.00 & 18.50 & 50 \\
	24. & Чистка обуви, мытье рук & 19.00 & 19.10 & 10 \\
	25. & У  ж  и  н & 19.10 & 19.30 & 20 \\
	26. &    Время для личных потребностей военнослужащих & 19.30 & 21.00 & 90 \\
	27. & Просмотр телепрограммы "ВРЕМЯ" & 21.00 & 21.30 & 30 \\
	28. & Вечерняя прогулка & 21.30 & 21.40 & 10 \\
	29. & Вечерняя поверка & 21.40 & 21.50 & 10 \\
	30. & Вечерний туалет & 21.50 & 22.00 & 10 \\
	31. & О  т  б  о  й & 22.00 \\
	\hline
\end{tabular}
\section{Военно-административное деление Российской Федерации}\label{Okruga}
\textbf{Западный военный округ} - в административных границах Республики Карелия, Республики Коми, Архангельской, Белгородской, Брянской, Владимирской, Вологодской, Воронежской, Ивановской, Калининградской, Калужской, Костромской, Курской, Ленинградской, Липецкой, Московской, Мурманской, Нижегородской, Новгородской, Орловской, Псковской, Рязанской, Смоленской, Тамбовской, Тверской, Тульской, Ярославской областей, г. Москвы, г. Санкт-Петербурга, Ненецкого автономного округа;

\textbf{Южный военный округ} - в административных границах Республики Адыгея, Республики Дагестан, Республики Ингушетия, Кабардино-Балкарской Республики, Республики Калмыкия, Карачаево-Черкесской Республики, Республики Северная Осетия - Алания, Чеченской Республики, Краснодарского, Ставропольского краев, Астраханской, Волгоградской и Ростовской областей;

\textbf{Центральный военный округ} - в административных границах Республики Алтай, Республики Башкортостан, Республики Марий Эл, Республики Мордовия, Республики Татарстан, Республики Тыва, Удмуртской Республики, Республики Хакасия, Чувашской Республики, Алтайского, Красноярского, Пермского краев, Иркутской, Кемеровской, Кировской, Курганской, Новосибирской, Омской, Оренбургской, Пензенской, Самарской, Саратовской, Свердловской, Томской, Тюменской, Ульяновской, Челябинской областей, Ханты-Мансийского автономного округа - Югры и Ямало-Ненецкого автономного округа;

\textbf{Восточный военный округ} - в административных границах Республики Бурятия, Республики Саха (Якутия), Забайкальского, Камчатского, Приморского, Хабаровского краев, Амурской, Магаданской, Сахалинской областей, Еврейской автономной области, Чукотского автономного округа.

\section{Штатная струкрура и вооружение мотострелковых войск}
\begin{center}
\begin{tabular}{| l | l|l |l|}
\hline
\multicolumn{4}{|c|}{Мотострелковый взвод БТР}\\ \hline
\multicolumn{2}{|c|}{Управление} & \multicolumn{2}{|c|}{Мотострелковое отделение}\\ \hline
Командир взвода &  АК-74, ПМ  &  Командир отделения & АК-74 \\ \hline
Зам. командира взвода & АК-74  & Оператор-пулеметчик & АкС-У\\ \hline
Снайпер & СВД &					 Водитель-БТР & АКС-у\\ 		   \hline
Пулеметчик ПК & ПК &			 Грантаметчик & 4     \\ \hline
Помощник пулеметчика & АК-74  &  Пом. гранатометчика & АК-74        \\ \hline
Стрелок-санитар & АК-74  &		 Пулеметчик & РПК-74    \\ \hline
\multicolumn{2}{|l|}{  } &   	 Старший стрелок) & АК-74 \\ \hline 	
\multicolumn{2}{|l|}{  } &   	 Стрелок-2 (3 мсо-1) & АК-74 \\ \hline 
\end{tabular}
\end{center}
\begin{center}
\begin{tabular}{| l | l|l |l|}
\hline
\multicolumn{4}{|c|}{Мотострелковый взвод БМП}\\ \hline
\multicolumn{2}{|c|}{Управление} & \multicolumn{2}{|c|}{Мотострелковое отделение}\\ \hline
Командир взвода &  АК-74, ПМ  &  Командир отделения & АК-74 \\ \hline
Зам. командира взвода & АК-74  & Наводчик-оператор & АкС-У\\ \hline
Снайпер & СВД &					 Механик Водитель & АКС-у\\ 		   \hline
Пулеметчик ПК & ПК &			 Гранатометчик & РПГ-74     \\ \hline
Помощник пулеметчика & АК-74  &  Пом. гранатометчика & АК-74        \\ \hline
Стрелок-санитар & АК-74  &		 Пулеметчик & РПК-74    \\ \hline
\multicolumn{2}{|l|}{  } &   	 Старший стрелок) & АК-74 \\ \hline 	
\multicolumn{2}{|l|}{  } &   	 Стрелок & АК-74 \\ \hline 
\end{tabular}
\end{center}

\section{Работа командира взвода (отделения) после получения боевой задачи (оценка обстановки)}

Бой командир взвода, как правило, организует на местности, а если это невозможно - в исходном районе по карте (схеме) или на макете местности. В этом случае боевые задачи отделениям и приданым средствам командир взвода уточняет на местности в период выдвижения их к рубежу перехода в атаку.

До получения задачи командир взвода организует подготовку вооружения и боевой техники. 

\textbf{Получив боевую задачу, командир взвода:}
\begin{enumerate}
 \item Уясняет ее;
 \item Оценивает обстановку;
 \item Принимает решение;
 \item Проводит рекогносцировку;
 \item Отдает боевой приказ;
 \item Организует взаимодействие, боевое обеспечение и управление;
 \item Проверяет подготовку личного состава, вооружения и боевой техники к бою;
\end{enumerate}
В установленное время докладывает командиру роты о готовности взвода к выполнению боевой задачи.

\textbf{Оценивая обстановку, командир взвода должен изучить:}
\begin{description}
	\item[*]состав, положение и возможный характер действий противника, места расположения его огневых средств;
	\item[*]состояние, обеспеченность и возможности взвода и приданных подразделений;
	\item[*]состав, положение, характер действий соседей и условия взаимодействия с ними;
	\item[*] местность, ее защитные и маскирующие свойства, выгодные подступы, заграждения и препятствия, условия наблюдения и ведения огня;
	\item[*]наиболее вероятные направления действий самолетов, вертолетов и других воздушных целей противника на малых и предельно малых высотах.
\end{description}
Кроме того, командир взвода учитывает время суток и состояние погоды.

\textbf{В результате оценки командир определяет:}
\begin{description}
\item[$\bullet$] какой силы противник ожидается перед фронтом действий взвода, его сильные и слабые стороны, возможное соотношение сил и средств;

\item[$\bullet$]боевой взвода, боевые задачи отделениям, распределение сил и средств;

\item[$\bullet$] на каком этапе боя и с кем из соседей поддерживать наиболее тесное взаимодействие;

\item[$\bullet$]порядок маскировки и использования защитных свойств местности.
\end{description}
\section{Работа командира взвода (отделения) после получения боевой задачи (рекогносцировка)}

Важным этапом в работе командира является \textbf{рекогносцировка}. 

При проведении рекогносцировки командир взвода на местности указывает ориентиры, положение противника (направление его действия), места расположения его огневых средств; 

Уточняет задачи отделениям и указывает места спешевания мотострелковых отделений (места позиций отделений, огневых позиций боевых машин пехоты, бронетранспортеров, танков и других огневых средств).


\section{\textcolor{red}{!!!Работа командира отделения (танка) при подготовке к боевым действия и в ходе боя.}}
\section{\textcolor{red}{!!!Обязанности командира отделения (танка) при подготовке к боевым действиям и в ходе боя.} }
\section{\textcolor{red}{!!!Виды инструктажей проводимых в воинской части.}}
Обучение личного состава технике безопасности при проведении всех видов работ производится как в процессе боевой подготовки, так и методом проведения инструктажей по технике безопасности. Личный состав должен проходить следующие виды инструктажа:
\begin{itemize}
\item вводный;
\item первичный на рабочем месте;
\item периодический на рабочем месте;
\item перед началом работ;
\item внеочередной.
\end{itemize}
\textbf{Вводный инструктаж }проводится с вновь прибывшим в воинскую часть личным составом независимо от его стажа работы и квалификации в целях ознакомления с правилами и мерами безопасности, действующими на территории части, а также с опасностями, которые могут возникать при эксплуатации техники связи и АСУ и использовании различного оборудования. Вводный инструктаж должен проводиться инженером по технике безопасности или начальником службы (подразделения) по инструкции, утвержденной командиром части.

\textbf{Первичный и периодический инструктажи }на рабочем месте проводятся командирами подразделений (командирами, начальниками, в введении которых находится техника связи и АСУ) с каждым в отдельности военнослужащим (рабочим, служащим РА), по специально разработанной программе, непосредственно на рабочем месте. Групповой инструктаж допускается только для личного состава одинаковой профессии и в пределах общего рабочего места.

\textbf{Первичный инструктаж} имеет целью ознакомить личный состав с особенностями работы на конкретном рабочем месте или при выполнении каких-либо видов работ, показать источники возможной опасности, а также методы и приёмы безопасной работы.

\textbf{Проведение вводного и первичного инструктажа} на рабочем месте отмечается в контрольном листе, заполняется на каждого военнослужащего. (Приложение № 9).

\textbf{Периодический инструктаж проводится} не реже 1 раза в 3 месяца и имеет цель закрепить и расширить первично полученные знания личного состава по правилам и мерам безопасности.

\textbf{Первичный и периодический} инструктажи должны содержать следующие основные вопросы:
\begin{itemize}
\item назначение и оборудование рабочего места;
\item опасности, которые могут возникать при производстве работ;
\item питающие напряжения на рабочем месте, способы их контроля и схема защиты;
\item правила поведения на данном рабочем месте;
\item порядок содержания и обслуживания рабочего места перед началом, во время и по окончании работ;
\item порядок взаимодействия с соседними рабочими местами;
\item правила поведения при аварийной обстановке;
\item назначение и правила пользования средствами защиты;
\item - оказание первой помощи пострадавшим при несчастных случаях.
\end{itemize}
Инструктаж перед началом работ проводится непосредственным начальником (ответственным руководителем работ) личного состава и имеет целью:

напомнить основные правила и приемы безопасного выполнения работ и охарактеризовать наиболее опасные операции предстоящей работы;

определить порядок использования защитных средств и взаимодействия личного состава и порядок действия в аварийных ситуациях.

Внеочередной инструктаж проводится командирами подразделений (командирами, начальниками, в введении которых находится техника связи и АСУ) в следующих случаях:
\begin{itemize}
\item при замене или модернизации техники связи и АСУ, если изменяются условия безопасности в работе;

\item при нарушении личным составом правил безопасности;

\item при несчастном случае, происшедшем в части.
\end{itemize}
Все виды инструктажей (кроме вводного и первичного инструктажа на рабочем месте) учитываются в журнале учета инструктажа личного состава по технике безопасности (Приложение № 10).

На основании типового перечня работ с повышенной опасностью (Приложение № 14) для каждой воинской части разрабатывается свой (более конкретный) перечень работ с повышенной опасностью, который утверждается командиром части. В каждой службе (подразделении), на каждом боевом посту должна быть выписка из утвержденного перечня работ с повышенной опасностью, составленная с учетом особенностей, характерных для данной службы (подразделения), боевого поста.


\section{Требования к безопасности при организации и несении боевого дежурства.}
\begin{center}
	\textbf{Обще требования безопасности при организации и несении  боевого дежурства. }
\end{center}
Личный, состав, привлекаемый на боевое дежурство, обязан:
\textbf{  Личный, состав, привлекаемый на боевое дежурство, обязан:}
  \begin{itemize}
 \item  знать и строго выполнять инструкции по требованиям безопасности, правилам пожарной безопасности для личного состава дежурных смен (расчетов);
 \item  проводить безопасную и безаварийную эксплуатацию вооружения и военной техники дежурных сил, а также сил, обеспечивающих боевое дежурство;
 \item  быть в готовности самостоятельно устранять неисправности вооружения и военной техники, входящие в его компетенцию;
 \item  уметь автономно выполнять задачи в аварийных и нештатных ситуациях, возникающих в ходе несения боевого дежурства.
\end{itemize}
\textbf{Военнослужащие, привлекаемые к несению боевого дежурства, должны:}
\begin{itemize}
	\item  соответствовать предъявляемым деловым и морально-психологическим требованиям;
	\item  пройти медицинское освидетельствование и быть годными по состоянию здоровья к выполнению, поставленных задач (кроме того, перед каждым заступлением на боевое дежурство - пройти медицинский осмотр); 
	\item  изучить и знать закрепленные вооружение и военную технику;
	\item  быть обученными безопасным методам выполнения функциональных обязанностей на своем рабочем месте, оказанию доврачебной медицинской помощи пострадавшим при несчастных случаях;
	\item  иметь допуск к самостоятельной работе, соответствующую должности : квалификационную группу по электробезопасности;
	\item пройти целевой инструктаж и знать последствия нарушений установленных требований безопасности. 
\end{itemize}

На боевое дежурство запрещается назначать военнослужащих, не приведенных к Военной присяге, не. усвоивших программу соответствующей подготовки в установленном объеме, совершивших проступки, по которым. ведется расследование, не прошедших медицинское освидетельствование и медицинский осмотр.
\section{Перечислите основное способы определения сторон горизонта по небесным светилам.}

При отсутствии компаса или в районах магнитных аномалий, где компас может дать ошибочные показания, стороны горизонта можно определить по небесным светилам: днём по Солнцу, а ночью – по Полярной звезде или Луне.

В северном полушарии (для наших широт: с.ш. $57^{o}$, в.д. $56^{o}$) Солнце примерно находится (зимой) в 7.00 – на востоке, в 13.00 – на юге, в 19.00 – на западе (летом 8.00, 14.00, 20.00 – соответственно). Положение солнца в эти часы и укажет соответственно направления на восток, юг и запад.

\textbf{Примечание}:
\begin{enumerate}
	\item Западно-Европейское время – время нулевого Гринвичского меридиана, к востоку (в 1-ом поясе «-1 час») – среднеевропейское время;
	\item Земной шар совершает оборот вокруг своей оси за 23 часа 56 мин 4 сек;
	\item В 1930 году в СССР принято декретное время – на 1 час вперед поясного времени;
	\item В 1981 году в СССР принято летнее время – на 1 час вперед декретного времени (примерно с 1 апреля по 1 октября);
	\item В дни весеннего и осеннего равноденствия Солнце восходит строго на востоке, а заходит строго на западе (21марта, 23 сентября).
\end{enumerate}
Для более точного определения сторон горизонта по Солнцу используются наручные часы. В горизонтальном положении они устанавливаются так, чтобы часовая стрелка была направлена на Солнце. Угол между часовой стрелкой и направлением на цифру 1 на циферблате часов делится пополам прямой линией, которая указывает направление на юг. До полудня надо делить пополам ту дугу, которую стрелка должна пройти до 13.00 зимой (до 14.00 летом), а после полудня – ту дугу, которую она прошла после 13.00 зимой (до 14.00 летом).
\begin{figure}[h]
\centering
\includegraphics[width=0.7\linewidth, height=0.2\textheight]{img/ClokSunStorona}
\caption[Определение старон света часами]{}
\label{fig:ClokSunStorona}
\end{figure}
Полярная звезда всегда находится на севере. Ночью на безоблачном небе её легко найти по созвездию Большой медведицы. Через две крайние звезды Большой Медведицы нужно мысленно провести прямую линию и отложить на ней пять раз отрезок, равный расстоянию между крайними звёздами. Конец пятого отрезка укажет положение Полярной звезды, которая находится в созвездии Малой Медведицы (конечная звезда малого ковша).

Полярная звезда может служить надёжным ориентиром для выдерживания направления движения, т.к. её положение на небосклоне с течением времени практически не изменяется. Точность определения направления по Полярной звезде составляет $2-3^{o}$.
\begin{figure}[h]
\centering
\includegraphics[width=0.7\linewidth]{img/PolarnayaZvezda}
\caption[Определение положения полярной звезды]{}
\label{fig:PolarnayaZvezda}
\end{figure}


\begin{enumerate}
	\item По  Луне стороны горизонта определяются следующим образом:
	\begin{enumerate}
\item  Разделить на глаз радиус диска Луны на 6 равных частей и оценить, сколько таких частей содержится в поперечнике видимого серпа Луны.
\item  Если Луна прибывает (видна правая часть диска), то полученное число надо вычесть из часа наблюдения, который следует предварительно заметить; при ущербе же Луны (видна левая часть лунного диска) указанное число прибавляют к часу наблюдения. Чтобы не спутать, когда брать сумму, а когда разность, можно пользоваться мнемоническим правилом: видимая левая часть лунного диска ассоциируется с буквой С – "сумма", а видимая правая часть лунного диска ассоциируется с буквой Р – "разность".
Полученная сумма или разность укажет час, когда в том направлении, где наблюдается Луна, будет находиться Солнце.
\item  Определив этот час и принимая Луну за Солнце, найти направление на юг, как это делается при ориентировании по Солнцу.
\end{enumerate}
\end{enumerate}
По Луне стороны горизонта определяются просто, когда виден весь её диск (полнолуние) или правая / левая половина диска Луны (см. таблицу \ref{tab:MoonFaz}).
% Таблица была сгенерированная в онлайн программе 
\begin{table}[]
	\centering
	\caption{Таблица фаз луны}
	\label{tab:MoonFaz}
	\begin{tabular}{|l|l|l|l|}
		\hline
		\multirow{2}{*}{Фаза луны}        & \multicolumn{3}{l|}{Время (зимой)} \\ \cline{2-4} 
		& 19.00      & 01.00      & 7.00     \\ \hline
		Видна правая половина диска луны  & юг         & запад      & ---      \\ \hline
		Полнолуние (виден весь диск Луны) & восток     & юг         & запад    \\ \hline
		Видна левая половина диска Луны   & ---        & восток     & юг       \\ \hline
	\end{tabular}
\end{table}
 
\section{Перечислите основные способы определения сторон горизонта по признакам местных предметов.}
\textbf{ Определение сторон горизонта по местным предметам.}
 Если нет компаса и не видно небесных светил, то стороны горизонта могут быть определены по признакам местных предметов. Из долголетних наблюдений установлено:
 \begin{itemize}
 	\item     мох или лишайник покрывают стволы деревьев, камни и пни с северной стороны, если мох растёт по всему стволу дерева, то на северной стороне, особенно у корня, его больше;
 	\item     кора деревьев с северной стороны обычно грубее и темнее, чем с южной;
 	\item      весной трава на северных окраинах лесных прогалин и полян, а также с южной стороны отдельных деревьев, пней, больших камней растёт гуще;
 	\item      муравейники, как правило, находятся к югу от ближайших деревьев и пней, южная сторона муравейника более пологая, чем северная;
 	\item      на южных склонах весной снег тает быстрее, чем на северных;
 	\item      просеки местных масштабов почти всегда прорубают строго по линии север-юг, восток-запад;
 	\item      на торцах столбов, устанавливаемых на перекрёстках просек, кварталы нумеруются с запада на восток (слева направо), цифры с меньшими номерами располагаются на северо-западе и северо-востоке, с большими - на юго-западе и юго-востоке;
 	\item      алтари православных церквей и часовен обращены на восток, а колокольни - к западу, алтари католических церквей (костелов) обращены на запад;
 	\item приподнятый конец нижней перекладины креста церквей обращен на север.
 \end{itemize}
\section{Назначение топографических карт.}
\textbf{\index{Топографические карты}} служат основным источником информации о местности и используются для ее изучения, определения расстояний и площадей, дирекционных углов, координат различных объектов и решения других измерительных задач. Они широко применяются при управлении войсками, а также в качестве основы для боевых графических документов и специальных карт. Топографические карты (преимущественно масштаба 1: 100000 и 1: 200000) служат основным средством ориентирования на марше и в бою.
\section{Подготовка данных для движения по азимуту.}
При изучении местности в направлении движения оценивают главным образом ее проходимость, маскировочные и защитные свойства, определяют труднопроходимые и непроходимые препятствия и пути их обхода.

Выбор маршрута и ориентиров. Начертание маршрута зависит от характера местности, наличия ориентиров на ней и от условий предстоящего движения. Основное требование к маршруту состоит в том, чтобы он обеспечивал быстрый, а в боевой обстановке и скрытный выход к указанному пункту.

Маршрут выбирают с таким расчетом, чтобы он был с минимальным числом поворотов. В маршрут включают дороги, просеки и другие линейные ориентиры, направление которых совпадает с направлением движения. Это облегчит выдерживание направлений движения. Точки поворота маршрута намечают у ориентиров, которые можно легко опознать на местности (например, постройки башенного типа, перекрестки дорог, мосты, путепроводы, геодезические знаки).
При выборе ориентиров на участках маршрута необходимо учитывать способ выдерживания направления движения и точность, которую он обеспечивает. Например, точность выдерживания направления движения по компасу составляет 0,1 пройденного расстояния. Если расстояние между ориентирами на участке маршрута будет 6 км, то при выходе к очередному ориентиру отклонение может быть 600 м. На отыскание ориентира на местности в этом случае потребуется много времени.

Опытным путем установлено, что расстояния между j ориентирами по маршруту движения не должны превышать 1—2 км при движении днем в пешем порядке, а при движении на машине и выдерживании направлений по гирополукомпасу—6—10 км. Для движения ночью ориентиры намечаются по маршруту чаще. Чтобы обеспечить скрытный выход к указанному пункту, маршрут намечают по лощинам, массивам растительности и другим объектам, обеспечивающим маскировку движения. Необходимо избегать передвижений по гребням возвышенностей и открытым участкам. Примерный вариант выбора маршрута показан на рис. \ref{fig:AzimuiForAwokr}.
\begin{figure}[h]
\centering
\includegraphics[width=0.7\linewidth]{img/AzimuiForAwokr}
\caption[Выбор маршрута для движения по азмимуту]{}
\label{fig:AzimuiForAwokr}
\end{figure}

\section{Виды кровотечений. Правила и способы временной остановки кровотечения.---}

\newpage
\section{Особенности остановки кровотечения.}{Особенности остановки кровотечения при ранениях шеи и травматических ампутациях}
\textbf{1. Ранения шеи,} сопровождающиеся артериальным наружным кровотечением, обычно приводят к смерти сразу после ранения. Необходимость в остановке кровотечения возникает лишь в исключительно редких случаях. Для этого рекомендуется освобожденное от оболочки содержимое перевязочного пакета прижать к кровоточащей ране.

Руку, противоположную стороне ранения, укладывают на голову пострадавшего так, чтобы плечо соприкасалось с боковой поверхностью головы и шеи, а предплечье лежало на своде черепа.

Таким образом, плечо раненого играет роль шины, предохраняющей от сдавления крупных сосудов шеи неповрежденной стороны рис(\ref{fig:HeadKillHand}). Жгут накладывается вокруг шеи и плеча раненого.


После того, как одним из необходимых способов наружное кровотечение остановлено, раненого, при возможности, целесообразно освободить от мокрой одежды и тепло укрыть.

Всех раненых с кровопотерей беспокоит жажда, поэтому им следует без ограничений давать пить воду, а если возможно, то теплый чай.
Кровотечения из незначительных ран шеи останавливают путём наложения повязки.

Повязка на шею рис.(\ref{fig:mumia})  накладывается круговым бинтованием. Для предупреждения ее соскальзывания вниз, круговые туры на шее комбинируют с турами крестообразной повязки на голове.

\begin{figure}[h]
	\centering
	\begin{subfigure}[b]{0.3\textwidth}
		\includegraphics[width=0.4\textwidth]{img/HeadKillHand}
		\caption{Наложение давящей повязки в области шеи}
		\label{fig:HeadKillHand}
	\end{subfigure}
	~ %add desired spacing between images, e. g. ~, \quad, \qquad, \hfill etc. 
	%(or a blank line to force the subfigure onto a new line)
	\begin{subfigure}[b]{0.3\textwidth}
		\includegraphics[width=0.3\textwidth]{img/mumia}
		\caption{Циркулярная повязка на шею}
		\label{fig:mumia}
	\end{subfigure}
	~ %add desired spacing between images, e. g. ~, \quad, \qquad, \hfill etc. 
	%(or a blank line to force the subfigure onto a new line)
	\caption{Изображения методов остановки кровотечения}\label{fig:tekutkakD}
\end{figure}


\textbf{2. Неотложная помощь при травматических ампутациях конечностей}

Прежде всего, необходимо остановить кровотечение из культи конечности или кисти наложением давящей повязки, надувными манжетами (жгут накладывается в крайнем случае). Вместо стандартного жгута кровоостанавливающего используют ремень, галстук, туго свернутый платок, косынку. Поврежденную конечность держать в возвышенном положении. необходимо уложить пострадавшего, дать ему обезболивающее средство, напоить крепким чаем. Раненую поверхность укрыть чистой или стерильной салфеткой.

\textbf{Техника наложения возвращающейся повязки.}

Бинтование начинают закрепляющими круговыми турами в верхней трети пострадавшего сегмента конечности. Затем удерживают бинт первым пальцем левой руки и делают перегиб на передней поверхности культи. Ход бинта ведут в продольном направлении через торцевую частью культи на заднюю поверхность. Каждый продольный ход бинта закрепляют круговым ходом. Выполняют перегиб бинта на задней поверхности культи ближе к торцевой части и ход бинта возвращают на переднюю поверхность. Каждый возвращающийся тур фиксируют спиральными ходами бинта от торцевой части культи.


Если культя имеет выраженную конусовидную форму, то повязка получается более прочной, когда второй возвращающийся ход бинта проходит перпендикулярно первому и перекрещивается на торце культи с первым возвращающимся туром под прямым углом. Третий возвращающийся ход следует проводить в промежутке между первым и вторым.

Возвращающиеся ходы бинта повторяют до тех пор, пока культя не будет надежно забинтована.

\textbf{Возвращающаяся повязка на культю предплечья }. Повязка начинается круговыми турами в нижней трети плеча, для предупреждения соскальзывания повязки. Затем ход бинта ведут на культю предплечья и накладывают возвращающуюся повязку. Бинтование завершают круговыми турами в нижней трети плеча \textbf{рис.(\ref{fig:VozvroshaUskultaoredolechya})}.

\begin{figure}[h]
\centering
\includegraphics[width=0.2\linewidth, height=0.1\textheight]{img/VozvroshaUskultaoredolechya}
\caption[]{Возвращающаяся повязка на культю предплечья}
\label{fig:VozvroshaUskultaoredolechya}
\end{figure}


\textbf{Возвращающаяся повязка на культю плеча.} Повязка начинается круговыми турами в верхней трети культи плеча. Затем накладывают возвращающуюся повязку, которую перед завершением укрепляют ходами колосовидной повязки на плечевой сустав. Завершают повязку круговыми турами в верхней трети плеча \textbf{ рис.(\ref{fig:KrovPlecho})}.

\begin{figure}[h!]
\centering
\includegraphics[width=0.2\linewidth, height=0.1\textheight]{img/KrovPlecho}
\caption[]{Возвращающаяся повязка на культю плеча}
\label{fig:KrovPlecho}
\end{figure}

\textbf{Возвращающаяся повязка на культю голени.} Повязка начинается круговыми турами в верхней трети голени. Затем накладывают возвращающуюся повязку, которую укрепляют восьмиобразными ходами повязки на коленный сустав. Завершают повязку круговыми турами в верхней трети голени.

\textbf{Возвращающаяся повязка на культю бедра.} Повязка начинается круговыми турами в верхней трети бедра. Затем накладывают возвращающуюся повязку, которую укрепляют ходами колосовидной повязки на тазобедренный сустав. Завершают повязку круговыми турами в области таза\textbf{ (рис.\ref{fig:KrovBedro})}.

\begin{figure}[h]
\centering
\includegraphics[width=0.2\linewidth, height=0.1\textheight]{img/KrovBedro}
\caption[Возвращающаяся повязка на культю бедра]{}
\label{fig:KrovBedro}
\end{figure}

Косыночная повязка на культю бедра. Середину косынки укладывают на торец культи, верхушку заворачивают на переднюю поверхность культи, а основание и концы косынки – на заднюю поверхность. Концы косынки обводят вокруг верхней трети бедра, формируя повязку, связывают на передней поверхности и фиксируют к узлу верхушку \textbf{рис.(\ref{fig:KrovKosinkaBedro})}

\begin{figure}[h!]
\centering
\includegraphics[width=0.2\linewidth, height=0.1\textheight]{img/KrovKosinkaBedro}
\caption[]{Косыночная повязка на культю бедра}
\label{fig:KrovKosinkaBedro}
\end{figure}

Аналогично накладываются косыночные повязки на культи плеча, предплечья и голени.
\\

\section{Виды и признаки обморожения и замерзания. Первая помощь при обморожении и замерзании.}
Различают четыре степени отморожения:
\begin{description}
	
	\item[Обморожение 1 степени –]  проявляется синюшностью, иногда характерной мраморностью кожи, болезненным зудом; после согревания отмечаются темно-синяя и багрово-красная окраска и отек кожи; заживление наступает через 3–4 дня;
	\item[Обморожение 2 степени –] кроме признаков, характерных для отморожения 1 степени, появляются пузыри, наполненные прозрачной желтоватой жидкостью или кровянистым содержимым;
	\item[Обморожение 3 степени –]  проявляется омертвением не только всех слоев кожи, но и глубже расположенных слоев мягких тканей;
	\item[Обморожение 4 степени –]  характеризуется омертвением всех мягких тканей, а также костей.
	О наступившем отморожении пострадавшие нередко узнают от встречных людей, которые замечают характерный белый (иногда синий) цвет кожи.
\end{description}

\textbf{Первая помощь при отморожении и замерзании}

Для профилактики отморожений необходимо следить за соответствием одежды и обуви погодным условиям. Одежда не должна значительно препятствовать движениям, обувь ни в коем случае не должна быть тесной, пропускающей влагу.
При длительном нахождении на улице в холодную погоду необходимо позаботиться о регулярном горячем питании, периодическом обогревании в теплом помещении или у костра. Лица, ранее перенесшие отморожения, у которых оно создает повышенную чувствительность к воздействию холода, должны уделять профилактике отморожений особое внимание.

При оказании первой помощи при обморожении нужно стремиться возможно быстрее восстановить кровообращение в отмороженном участке тела.
При легком отморожении достаточно растереть кожу ладонью или какой-либо тканью. Не следует растирать кожу снегом, так как его мелкие кристаллы легко повреждают измененные ткани, что может привести к их инфицированию. После покраснения кожи желательно протереть ее спиртом, водкой или одеколоном и укутать отмороженный участок.

Отогревать пострадавшего лучше в теплом помещении. При отморожении конечности ее погружают в теплую воду температурой около 20 градусов, которую постепенно (в течение 20 мин) повышают до 37–40 градусов. Кожу осторожно массируют по направлению от пальцев к туловищу (при наличии пузырей массаж делать нельзя), осторожно обмывают и просушивают тампоном, смоченным водкой или спиртом, накладывают стерильную повязку. Не нужно смазывать кожу «зеленкой», йодом или каким-либо жиром.

При общем замерзании пострадавших отогревают в теплой ванне (температура воды не выше 37  градусов), дают им внутрь (если сознание пострадавшего отсутствует, осторожно вливают) немного алкоголя, теплый чай или кофе, растирают тело, начиная от участков, наиболее пострадавших от холода. В тех случаях, когда поместить пострадавшего в ванну невозможно, его укладывают в постель, тело протирают спиртом, водкой или одеколоном, на отмороженные участки накладывают стерильные повязки, ногам придают возвышенное положение, поверх одеяла кладут грелки.
Когда поместить пострадавшего в тепло нельзя, следует обогреть его у костра и растереть кожу. В случае невозможности развести огонь нужно делать растирание на морозе, укрыв пострадавшего дополнительной одеждой. При отморожении лица нужно придать пострадавшему лежачее положение с низко опущенной головой.

При отсутствии дыхания и сердечной деятельности необходимо, продолжая общий массаж тела, немедленно приступить к искусственной вентиляции легких и наружному массажу сердца. Восстановление жизненных функций сопровождается постепенной нормализацией окраски кожного покрова, появлением сердечных сокращений и пульса, дыхания. У пострадавших наступает глубокий сон.

В случае тяжелого отморожения пострадавшего нужно срочно отправить в лечебное учреждение для осуществления медикаментозного и других видов лечения.
\newpage
\section{Первая  помощь при поражении током.}
%\begin{wrapfigure}{l}{0.4\textwidth}
%	\includegraphics[width=0.9\linewidth, height=0.2\textheight]{img/TOKOMebnut1}
%	\caption{картинка}
%	\label{ris:image}
%\end{wrapfigure}
%Обеспечь свою безопасность. Надень сухие перчатки (резиновые, шерстяные, кожаные и т.п.), резиновые сапоги. По возможности отключи источник тока. При подходе к пострадавшему по земле иди мелкими, не более 10 см, шагами. 

\begin{SCfigure}[][h!]
	\centering
	\caption{Обеспечь свою безопасность. Надень сухие перчатки (резиновые, шерстяные, кожаные и т.п.), резиновые сапоги. По возможности отключи источник тока. При подходе к пострадавшему по земле иди мелкими, не более 10 см, шагами. }
	\includegraphics[width=0.4\linewidth, height=0.15\textheight]%
	{img/TOKOMebnut1}% picture filename
\end{SCfigure}
{}
\begin{SCfigure}[][h!]
	\centering
	\caption{Сбрось с пострадавшего провод сухим токонепроводящим предметом (палка, пластик). Оттащи пострадавшего за одежду не менее чем на 10 метров от места касания проводом земли или от оборудования, находящегося под напряжением. }
	\includegraphics[width=0.4\linewidth, height=0.15\textheight]%
	{img/TOKOMebnut2}% picture filename
\end{SCfigure}
{}
\begin{SCfigure}[][h!]
	\centering
	\caption{Определи наличие пульса на сонной артерии, реакции зрачков на свет, самостоятельного дыхания.}
	\includegraphics[width=0.4\linewidth, height=0.15\textheight]%
	{img/TOKOMebnut3}% picture filename
\end{SCfigure}
{}
\begin{SCfigure}[][h!]
	\centering
	\caption{При отсутствии признаков жизни проведи сердечно-легочную реанимацию}
	\includegraphics[width=0.4\linewidth, height=0.15\textheight]%
	{img/TOKOMebnut4}% picture filename
\end{SCfigure}
{}
\begin{SCfigure}[][h!]
	\centering
	\caption {При восстановлении самостоятельного дыхания и сердцебиения придай пострадавшему устойчивое боковое положение. }
	\includegraphics[width=0.4\linewidth, height=0.15\textheight]%
	{img/TOKOMebnut5}% picture filename
\end{SCfigure}
{}
\begin{SCfigure}[][h!]
	\centering
	\caption{Если пострадавший пришел в сознание, укрой и согрей его. Следи за его состоянием до прибытия медицинского персонала, может наступить повторная остановка сердца. }
	\includegraphics[width=0.4\linewidth, height=0.15\textheight]%
	{img/TOKOMebnut6}% picture filename
\end{SCfigure}

\newpage
\section{\textcolor{green}{(Доработать потом )}Первая  помощь при утоплении, обвалах.}
\textbf{Первая помощь при утоплении}

Утопление обычно наблюдается в результате пренебрежения правилами купания. Причина-ми утопления могут быть неумение плавать, недомогание, переутомление, предшествующее перегревание, алкогольное опьянение, испуг находящегося в воде человека. Иногда тонут из-за переоценки своих возможностей даже хорошие пловцы. Утопление имеет место при форсировании водных преград, стихийных бедствиях, связанных с наводнениями и большим подъемом воды.

При спасении утопающего в первую очередь следует позаботиться о собственной безопасности. Для утопающего характерны судорожные, не всегда достаточно осознанные движения, которые могут представлять серьезную опасность для спасателя.
Подплывать к утопающему следует сзади и, схватив его за волосы или подмышки, перевернуть лицом вверх таким образом, чтобы оно было над водой. Пострадавшего нужно как можно быстрее вытащить из воды, освободить от затрудняющей дыхание одежды (расстегнуть воротник, поясной ремень и др.).

После этого спасатель укладывает пострадавшего животом на бедро своей согнутой в коле-не ноги лицом вниз, чтобы голова пострадавшего находилась ниже туловища, очищает полость рта от ила, песка, слизи. Затем энергичным надавливанием на корпус освобождает легкие и желудок от воды. На очищение дыхательных путей и их освобождение от воды следует тратить не более 20–30 с.

Удаление воды из дыхательных путей Если у пострадавшего отсутствует дыхание, необходимо, не теряя ни минуты, начинать реанимационные мероприятия.
Восстановить жизнедеятельность пострадавшего можно, если человек пробыл под водой не более 5 мин, и ему немедленно была оказана помощь. Однако наблюдаются случаи, когда из-за спазма гор-тани легкие не заполняются водой, а сердце при этом еще некоторое время продолжает работать. В этих случаях спасение возможно даже после полу-часового пребывания человека под водой.
Следует помнить, что искусственное дыхание и за-крытый массаж сердца являются лишь первоочередными мероприятиями.
Для определения тяжести состояния и дальнейшего лечения необходимо без промедления вызвать врача и по возможности быстро транспортировать пострадавшего в лечебное учреждение, где должны быть продолжены реанимационные мероприятия в полном объеме.

\textbf{Первая помощь при обвалах	}

Пострадавшие, оказавшиеся под завалами зданий, оборонительных сооружений и т.п., могут иметь различные повреждения, а также находиться в состоянии острой гипоксии от удушья, вызванного закупоркой дыхательных путей пылью, землей, недостатком, воздуха, сдавлением груди и шеи.
После осторожного извлечения пострадавшего из-под обвала ему очищают рот и нос и, при необходимости, производят реанимационные мероприятия. После восстановления у пострадавшего самостоятельного дыхания при необходимости проводят противошоковые мероприятия, наложение повязки, иммобилизацию переломов, а затем – эвакуация в лечебное учреждение.

Особое внимание обращают на выявление факта длительного сдавления пострадавшего. Своеобразный комплекс расстройств, называемый синдромом сдавления, возникает и разви-вается в результате продолжительного (свыше 3 часов) сдавления мягких тканей - чаще ниж-них конечностей. Этот синдром развивается после возобновления кровообращения при освобождении от длительного сдавления тканей. Тяжесть состояния пострадавших зависит от обширности повреждения мягких тканей и длительности нахождения под обломками завалов. На конечностях, подвергшихся длительному сдавлению, наблюдается бледность, иногда синюшные пятна. Общее состояние пострадавших вначале обычно не вызывает опасений. Однако через несколько часов появляется синюшно-багровая окраска конечности, на коже возникают пузыри, наполненные кровянистым содержимым. В последующем отмечается омертвение тканей. Всасывание ядовитых продуктов распада поврежденных тканей приводит к резкому ухудшению общего состояния пострадавших, особенно существенно снижается функция почек. 

Можно полное прекращение выделения мочи.

В случае установления признаков длительного сдавления пострадавших рассматривают как тяжелопораженных независимо от их состояния. Оказание им медицинской помощи начинается с быстрого устранения сдавления, тугового бинтования (от стопы) и транспортной им-мобилизации поврежденной конечности. Необходимо ввести анальгетик из шприц-тюбика. При тяжелых повреждениях конечности – накладывают жгут.

\section{Назначение, боевые свойства, устройство автомата Калашникова АК-74.}
Назначение 5,45-мм автомата Калашникова АК74

5,45-мм автомат Калашникова АК74 является индивидуальным оружием и предназначен для уничтожения живой силы и поражения огневых средств противника. Для поражения противника в рукопашном бою к автомату присоединяется штык-нож.

Боевые свойства 5,45-мм автомата Калашникова АК74

Для стрельбы из автомата применяются патроны с обыкновенными (со стальным сердечником) и трассирующими пулями.

Из автомата ведется автоматический или одиночный огонь. Автоматический огонь является основным видом огня: он ведется короткими (до 5 выстрелов) и длинными (до 10 выстрелов) очередями и непрерывно. Подача патронов при стрельбе производится из коробчатого магазина емкостью 30 патронов. Магазины автомата взаимозаменяемы.

Прицельная дальность стрельбы – 1000 м. Наиболее действительный огонь по наземным целям, по самолетам, вертолетам и парашютистам – на дальности до 500 м. Сосредоточенный огонь по наземным групповым целям ведется на дальность до 1000 м.

Дальность прямого выстрела: по грудной фигуре – 440 м, по бегущей фигуре – 625 м.

Темп стрельбы около 600 выстрелов в минуту.

Боевая скорострельность: при стрельбе очередями – до 100 выстрелов в минуту; при стрельбе одиночными выстрелами – до 40 выстрелов в минуту.

Вес автомата без штыка-ножа со снаряженным патронами пластмассовым магазином: АК74 – 3,6 кг. Вес штыка-ножа с ножнами – 490 г.

Основные характеристики автомата Калашникова, можно увидеть в таблице(\ref{tabular:AK74})

\newpage
\begin{table}[h!]
\caption{Сводная таблица баллистических и конструктивных данных АК74}
\label{tabular:AK74}
\begin{center}
\begin{tabular}{|l|p{11cm}|l|}
	\hline
	\rowcolor[gray]{.9} № & Наименование данных &  АК74\\
	 1&		Прицельная дальность,&       м	1000\\
	 2&		Дальность прямого выстрела:&	 \\
	  &   	по грудной фигуре, м	&440\\
	  &  	по бегущей фигуре, м	&625\\
	 3&		Темп стрельбы, выстрелов в минуту	&$\backsim$600\\
	 4&		Боевая скорострельность, выстрелов в минуту:&	 \\
	  &		при стрельбе одиночными выстрелами	&40\\
	  &	 	при стрельбе очередями	&100\\
	 5&		Наиболее действительный огонь по наземным целям, м	&до 500\\
	 6&		Наиболее действительный огонь  по самолетам, вертолетам и парашютистам, м	&до 500\\
	 7&		Сосредоточенный огонь по наземным групповым целям, м	&до 1000\\
	 8&		Начальная скорость пули, м/сек	&900\\
	 9&	    Дальность, до которой сохраняется убойное действие пули,м &	1350\\
	 10&	Предельная дальность полета пули,м &	3150\\
	 11&	Вес автомата (без штык-ножа), кг &	 \\
	   &	 с неснаряженным пластмассовым магазином	&3,3\\
	   &	 со снаряженным пластмассовым магазином	&3,6\\
	 12&	Емкость магазина, патронов	&30\\
	 13&    Вес магазина, кг	& \\
	   &     пластмассового для АК74	&0,23\\
	 14&	Вес штыка-ножа, кг:&	 \\
	   &      с ножнами	&0,49\\
	   &    без ножен	&0,32\\
	 15&	Калибр, мм	&5,45\\
	 16&	Длина автомата, мм:	& \\
	   &	автомата с примкнутым штыком-ножом	&1089\\
	   &	автомата без штыка-ножа	&940\\\
	 17&	Длина ствола, мм	&415\\
	 18&	Длина нарезной части ствола, мм	&372\\
	 19&	Число нарезов, шт.	&4\\
	 20&	Длина хода нарезов, мм	&200\\
	 21&    Длина прицельной линии, мм	&379\\
	 22&	Толщина мушки, мм	&2\\
	 23&	Вес патрона, г	&10,2\\
	 24&	Вес пули со стальным сердечником, г&	3,4\\
	 25&	Вес порохового заряда, г	&1,45\\
	 \hline
\end{tabular}
\end{center}
\end{table}
\newpage

\section{Назначение, боевые свойства, устройство грантамето РПГ-7.}

Ручной противотанковый гранатомет РПГ-7В предназначен для борьбы с танками, самоходно-артиллерийскими установками и другими бронированными средствами противника, а также для уничтожения живой силы противника, находящейся в легких укрытиях и сооружениях городского типа.

Стрельба производится выстрелами с надкалиберной противотанковой, осколочной и термобарической гранатами.

Калибр – 40 мм

Вес с оптическим прицелом – 6,7 кг
Боевые свойства гранатомета и характеристики выстрелов можно посмотреть в  таб.(\ref{tab:RPG-7}) и таб.(\ref{tab:bumbumgrante} стр. (\pageref{tab:bumbumgrante})).

Гранатомет состоит из следующих частей и механизмов:
\begin{description}
	\item[---]ствола с механическим (открытым) прицелом;
	\item[---]ударно-спускового механизма (УСМ) с предохранителем;
	\item[---]бойкового механизма;
	\item[---]оптического прицела.
\end{description}
\begin{table}[h!]
	\centering
	\caption{ТТХ РПГ-7}
	\label{tab:RPG-7}
	\begin{tabular}{|l|l|}
		\hline
		\multicolumn{2}{|l|}{Массо-габаритные характеристики} \\ \hline
		Калибр, мм                            & 40            \\ \hline
		Масса,кг                              & 3,3           \\ \hline
		Длина,мм                              & 950           \\ \hline
		\multicolumn{2}{|l|}{Боевые характеристики}           \\ \hline
		Дальность прямого выстрела,м          & до 330        \\ \hline
		Прицельная дальность,м                & до 700        \\ \hline
		Брнепробиваемость,мм                  & до 750        \\ \hline
		Масса гранаты,кг                      & 2-4,5         \\ \hline
		Начальная скорость гранаты, м/с       & 112-145       \\ \hline
		Калибр                                & 40-105        \\ \hline
	\end{tabular}
\end{table}

\newpage
\begin{table}[h!]
	\centering
	\caption{Характеристики гранатометных выстрелов }
	\label{tab:bumbumgrante}
		\begin{sideways}
	\begin{tabular}{|p{2cm}| p{3cm}| p{1.8cm}| p{3cm} |p{2.5cm} |p{3cm} |p{2cm} |}
		\hline
		Индекс выстрела      & Тип боевой части        & Масса выстрела & Калибр головной части, мм & Броне пробиваемость                  & Начальная скорость гранаты, м/с & Дальность, м \\
		ПГ-7В / 7П1          & куммулитивная           & 2,2            & 85                        & 260                                 & 120                             & 500                      \\
		ПГ-7ВМ / 7П6         & куммулитивная           & 2              & 70                        & 300                                 & 120                             & 500                      \\
		ПГ-7ВС / 7П13        & куммулитивная           & 2              & 72                        & 400                                 & 120                             & 500                      \\
		ПГ-7ВЛ Луч / 7П16    & куммулитивная           & 2,6            & 50093                     & 500                                 & 120                             & 500                      \\
		ПГ-7ВР Резюме        & тандемная куммулитивная & 4,5            & 64/105                    & ДЗ+650                              & 100                             & 200                      \\
		ОГ-7В Осколок / 7П50 & оскольчная              & 2              & 40                        & н/д площадь поражения оскольков 150 & 120                             & 700                      \\
		ТБГ-7В Танин / 7П33  & термобарическая         & 4.5            & 105                       & радиус поражения живой силы:10 м    & 100                             & 200                     \\
		\hline
	\end{tabular}
\end{sideways}
\end{table}



\newpage

\section{Назначение, боевые свойства, устройство пулемета ПКМ.}
\textbf{7,62-мм пулемет Калашникова} (ПК, ПКС — на станке, ПКБ — бронетранспортерный, ПКТ — танковый) является мощным автоматическим оружием и предназначен для уничтожения живой силы и огневых средств противника. Пулеметы ПК и ПКС также предназначены для поражения воздушных целей.

\textbf{Тактико-технические характеристики (ТТХ) пулемёта Калашникова (ПК, ПКС, ПКТ):}
\begin{itemize}
\item Для стрельбы из пулемета применяются патроны с обыкновенными, трассирующими и бронебойно-зажигательными пулями.
\item Стрельба из пулемета ведется короткими (до 10 выстрелов) и длинными (до 30 выстрелов) очередями и непрерывно.
\item Подача патронов в приемник при стрельбе производится из металлической ленты, уложенной в коробку. Емкость ленты — 100, 200 или 250 патронов.
\item Наиболее действительный огонь из пулемета по наземным и воздушным целям — на расстояния до 1000 м. Прицельная дальность стрельбы пулеметов ПК, ПКБ и ПКС—1500 м.
\item Дальность прямого выстрела по грудной фигуре — 400 м, а по бегущей фигуре — 650 м.
\item Темп стрельбы (техническая скорострельность) — около 650 выстрелов в минуту (пулемета ПКТ — 700 — 800 выстрелов в минуту).
\item Боевая скорострельность — до 250 выстрелов в минуту.
\item Охлаждение ствола пулемета воздушное, допускающее ведение непрерывного огня до 500 выстрелов, после чего при необходимости продолжения стрельбы нагретый ствол должен быть заменен запасным.
\item Стрельба из пулемета производится с сошки или с треножного станка конструкции Саможенкова. Станок обеспечивает ведение огня из пулемета по наземным и воздушным целям и повышает действительность стрельбы на предельных дальностях.
\item Угол горизонтального обстрела по наземным целям с применением ограничителей — около 90 градусов, а по воздушным целям — 360 градусов.
\item Высота линии огня при стрельбе со станка из положения лежа —320 мм, с колена —820 мм и сидя — 580 мм.
\item Пулемет ПКТ спаренный с пушкой имеет угол горизонтального обстрела 360 градусов .
\item Массовые данные:
\begin{itemize}
 \item пулемета ПК — 9 кг; 
 \item пулемета ПКС—16,7 кг; 
 \item пулемета ПКТ—10,5 кг;
 
 коробки с лентой и
 \begin{itemize}
  \item 100 патронами —3,9 кг,   
  \item с 200 патронами —8 кг, 
  \item с 250 патронами — 9,4 кг.
   \end{itemize}
\end{itemize}
\end{itemize}
\begin{figure}
\centering
\includegraphics[width=0.9\linewidth, height=0.2\textheight]{img/pk2a}
\caption{Основные части и механизмы пулемёта:}{а – пулемёта ПК; ; 1 – шомпол; 2 – затвор; 3 – затворная рама с извлекателем и газовым поршнем; 4 – возвратная пружина; 5 – направляющий стержень; 6 –ствольная коробка с крышкой, основанием приёмника и прикладом; 7 – ствол; 8 –трубка газового поршня с сошкой (у пулемёта ПК); 9 – коробка с лентой; 10 –принадлежность; 11 – спусковой механизм; 12 – электроспуск}
\label{fig:pk2a}
\end{figure}
\begin{figure}[h]
\centering
\includegraphics[width=0.7\linewidth, height=0.2\textheight]{img/pk2b}
\caption{Основные части и механизмы пулемёта:}{б – пулемёта ПКТ; 1 – шомпол; 2 – затвор; 3 – затворная рама с извлекателем и газовым поршнем; 4 – возвратная пружина; 5 – направляющий стержень; 6 –ствольная коробка с крышкой, основанием приёмника и прикладом; 7 – ствол; 8 –трубка газового поршня с сошкой (у пулемёта ПК); 9 – коробка с лентой; 10 –принадлежность; 11 – спусковой механизм; 12 – электроспуск}
\label{fig:pk2b}
\end{figure}


\section{Назначение, боевые свойства, устройство снайперской винтовки СВД}
\textbf{1.} 7,62-мм снайперская винтовка Драгунова (рис. \ref{fig:svd1}) является оружием снайпера и предназначена для уничтожения различных появляющихся, движущихся, открытых и маскированных одиночных целей.

Огонь из снайперской винтовки наиболее эффективен на расстояния до 800 м. Прицельная дальность стрельбы с оптическим 1300 м, с открытым прицелом – 1200 м.

Дальность прямого выстрела по грудной фигуре 430 м, а по бегущей фигуре – 640 м.

Боевая скорострельность до 30 выстрелов в минуту.

Масса снайперской винтовки без штыка-ножа с оптическим прицелом, неснаряженным магазином и щекой приклада 4,3 кг.
\begin{figure}[h!]
\centering
\includegraphics[width=0.7\linewidth, height=0.1\textheight]{img/svd1}
\caption{Общий вид снайперской винтовки Драгунова}{}
\label{fig:svd1}
\end{figure}

\textbf{2.} Для стрельбы из снайперской винтовки применяются винтовочные патроны с обыкновенными, трассирующими и бронебойно-зажигательными пулями или винтовочные снайперские патроны.

Огонь из снайперской винтовки ведется одиночными выстрелами.

Подача патронов при стрельбе производится из коробчатого магазина емкостью на 10 патронов.

\textbf{Основные части и механизмы снайперской винтовки, их работа при стрельбе}

\textbf{3.} Снайперская винтовка состоит из следующих основных частей и механизмов (рис. \ref{fig:svd2}):
\begin{description}
\item[---] ствола со ствольной коробкой, открытым прицелом и прикладом;
\item[---]крышки ствольной коробки;
\item[---]возвратного механизма;
\item[---] затворной рамы;
\item[---]затвора;
\item[---]газовой трубки с регулятором, газового поршня и толкателя е пружиной;
\item[---] ствольных накладок (правой и левой):
\item[---] ударно-спускового механизма;
\item[---]предохранителя;
\item[---] магазина;
\item[---] щеки приклада;
\item[---] оптического прицела;
\item[---] штыка-ножа.
\end{description}

В комплект снайперской винтовки входят: принадлежность, ремень, чехол для оптического прицела, сумка для переноски оптического прицела и магазинов, сумочка для переноски зимнего устройства освещения сетки, запасных батареек и масленки.

\textbf{4. }Снайперская винтовка является самозарядным оружием. Перезаряжание винтовки основано на использовании энергии пороховых газов, отводимых из канала ствола к газовому поршню

\begin{figure}
\centering
\includegraphics[width=0.8\linewidth, height=0.2\textheight]{img/svd2}
\caption{Основные части и механизмы снайперской винтовки:}{1 – штык нож; 2 – газовая трубка с регулятором; 3 – газовый поршень; 4 – толкатель с пружиной; 5 – возвратныймеханизм; 6 – крышка ствольной коробки; 7 – щека приклада; 8 – оптическийприцел; 9 – ударно спусковой механизм; 10 – магазин; 11 – предохранитель; 12 – стволсо ствольной коробкой, открытым прицелом и прикладом; 13 – ствольные накладки; 14 – затвор; 15 – затворная рама}
\label{fig:svd2}
\end{figure}

При выстреле часть пороховых газов, следующих за пулей, устремляется через газоотводное отверстие в стенке ствола в газовую камеру, давит на переднюю стенку газового поршня и отбрасывает поршень с толкателем, а вместе с ними и затворную раму в заднее положение. При отходе затворной рамы назад затвор открывает канал ствола, извлекает из патронника гильзу и выбрасывает ее из ствольной коробки наружу, а затворная рама сжимает возвратные пружины и взводит курок (ставит его на взвод автоспуска).

В переднее положение затворная рама с затвором возвращается под действием возвратного механизма, затвор при этом досылает очередной патрон из магазина в патронник и закрывает канал ствола, а затворная рама выводит шептало автоспуска из-под взвода автоспуска курка. Курок становится на боевой взвод. Запирание затвора осуществляется его поворотом влево и захождением боевых выступов затвора в вырезы ствольной коробки.

Для производства очередного выстрела необходимо отпустить спусковой крючок и нажать на него снова. После освобождения спускового крючка тяга продвигается вперед и ее зацеп заскакивает за шептало, а при нажатии на спусковой крючок зацеп тяги поворачивает шептало и разъединяет его с боевым взводом курка.

При выстреле последним патроном, когда затвор отойдет назад, подаватель магазина поднимает вверх останов затвора, затвор упирается в него и затворная рама останавливается в заднем положении. Это является сигналом о том, что надо снова зарядить винтовку.
\section{Назначение, боевые свойства, устройство гранат Ф-1.}
Назначение, боевые свойства и общее устройство ручной осколочной гранаты \index{Ф-1}

Ручная осколочная граната Ф-1 - граната дистанционного действия, предназначенная для поражения живой силы преимущественно в оборонительном бою.

Ручная оборонительная граната Ф-1 («лимонка») была разработана на основе французской осколочной гранаты F-1 модели 1915 г., отсюда обозначение Ф-1. Эту гранату не следует путать с современной французской моделью F1 с пластиковым корпусом и полуготовыми осколками и английской гранаты системы Лемона (с терочным запалом), поставлявшейся в Россию в годы первой мировой войны. На вооружение РККА граната Ф-1 принята с дистанционным взрывателем (запалом) Ковешникова. С 1941 г. вместо запала Ковешникова в гранате Ф-1 стал применяться более простой в изготовлении и обращении запал УЗРГ системы Е.М. Вицени.
\begin{figure}[h]
\centering
\includegraphics[width=0.2\linewidth, height=0.2\textheight]{img/f1}
\caption{Граната Ф-1 }
1 – корпус; 2 – разрывной заряд; 3 - запал
\label{fig:f1}
\end{figure}

Корпус гранаты при разрыве дает 290 крупных тяжелых осколков с начальной скоростью разлета около 730 м/с.
На образование убойных осколков идет 38% массы корпуса, остальное осколки попросту распыляется. Площадь разлета осколков - 75-82 м2.

Ручная осколочная граната Ф-1 состоит из корпуса (1), разрывного заряда (2) и запала (3). Корпус гранаты служит для помещения разрывного заряда и запала, а также для образования осколков при взрыве гранаты. Корпус гранаты чугунный, с продольными и поперечными бороздами, по которым граната обычно разрывается на осколки. В верхней части корпуса имеется нарезное отверстие для ввинчивания запала. При хранении, транспортировании и переноске гранаты в это отверстие ввернута пластмассовая пробка.

Разрывной заряд заполняет корпус и служит для разрыва гранаты на осколки. Запал гранаты предназначается для взрыва разрывного заряда гранаты.

Ручные осколочные гранаты Ф-1 комплектуется модернизированным унифицированным запалом к ручным гранатам (УЗРГМ). Капсюль запала воспламеняется в момент броска гранаты, а взрыв ее происходит через 3,2 - 4,2 с после броска. Граната безотказно взрываются при падении в грязь, снег, воду и т.п.

Метать гранату можно из различных положений и только из-за укрытия, из бронетранспортера или танка (самоходно-артиллерийской установки).

Характеристики	Граната Ф-1
Масса гранаты, г.	600
Масса боевого заряда, г.	60
Дальность броска, м.	35-45
Время замедления, с.	3,2-4,2
Радиус убойного действия осколков, м.	200
\begin{table}[h!]
	\caption{Боевые свойства оборонительной гранаты Ф-1}
	\label{tabular:F1}
	\begin{center}
		\begin{tabular}{|l|l|}
			\hline
			\rowcolor[gray]{.9} Характеристики &  Граната Ф-1\\
			Масса гранаты, г.	& 600\\
			Масса боевого заряда, г.	& 60\\
			Дальность броска, м.	& 35-45\\
			Время замедления, с.	& 3,2-4,2\\
			Радиус убойного действия осколков, м.	& 200\\
			\hline
		\end{tabular}
	\end{center}
\end{table}


\section{Назначение, боевые свойства, устройство гранат РГД-5.}
\textbf{Назначение, боевые свойства и общее устройство ручной осколочной гранаты РГД-5}

Ручная осколочная граната РГД-5 - граната дистанционного действия, предназначенная для поражения живой силы противника в наступлении и в обороне.

Площадь рассеивания осколков граната РГД-5 - 28-32 м2. Метание гранаты осуществляется из различных положений при действиях в пешем порядке и на бронетранспортере (автомобиле).

\begin{figure}[h]
\centering
\includegraphics[width=0.2\linewidth, height=0.2\textheight]{img/RGD5}
\caption{Граната РГД-5}
1 – корпус; 2 – запал; 3 – разрывной заряд; 4 – колпак; 5 – вкладыш колпака; 6 – трубка для запала; 7 – манжета; 8 – поддон; 9 – вкладыш поддона
\label{fig:RGD5}
\end{figure}

Граната РГД-5 (рис.\ref{fig:RGD5}) состоит из корпуса с трубкой для запала, разрывного заряда и запала (2) УЗРГМ (УЗРГМ-2). Кроме УЗРГМ и УЗРГМ-2 в боевых условиях могут применяться оставшиеся в войсках старые запалы УЗРГ, но они запрещены к применению при обучении.

Корпус гранаты (1) служит для помещения разрывного заряда (3), трубки для запала (6), а также для образования осколков при взрыве гранаты. Корпус состоит из двух частей - верхней и нижней. Верхняя часть корпуса состоит из внешней оболочки, называемой колпаком (4), и вкладыша колпака (5). К верхней части с помощью манжеты (7) присоединяется трубка для запала.

Трубка служит для присоединения запала к гранате и для герметизации разрывного заряда в корпусе.
Для предохранения трубки от загрязнения в нее ввинчивается пластмассовая пробка. При подготовке гранаты к метанию вместо пробки в трубку ввинчивается запал. 

Нижняя часть корпуса состоит из внешней оболочки, называемой поддоном (8), и вкладыша поддона (9). Разрывной заряд заполняет корпус и служит для разрыва гранаты на осколки.
Граната безотказно взрываются при падении в грязь, снег, воду и т.п.

\begin{table}[h!]
	\caption{Боевые свойства оборонительной гранаты РГД-5}
	\label{tabular:RGD5}
	\begin{center}
		\begin{tabular}{|l|l|}
			\hline
			\rowcolor[gray]{.9} Характеристики &  Граната РГД-5\\
			Масса гранаты, г.	& 310\\
			Масса боевого заряда, г.	& ???\\
			Дальность броска, м.	& 40-50\\
			Время замедления, с.	& 3,2-4,2\\
			Радиус убойного действия осколков, м.	& 25\\
			\hline
		\end{tabular}
	\end{center}
\end{table}

\section{Назначение, боевые свойства, устройство гранат РГО.}
\textbf{Назначение, боевые свойства и общее устройство ручной осколочной гранаты \index{РГО}}

Ручная осколочная граната РГО предназначенная для поражения живой силы преимущественно в оборонительном бою.

Ручная осколочная граната РГО (оборонительная) разработана на предприятии «Базальт» в конце 1970-х годов. Существенное отличие от аналогичных образцов заключается в оснащении ее датчиком цели и срабатывании при ударе о любую преграду.

\begin{figure}[h]
\centering
\includegraphics[width=0.2\linewidth, height=0.2\textheight]{img/RGO}
\caption[Граната РГО]{}
\label{fig:RGO}
\end{figure}
Граната состоит из корпуса, заряда взрывчатой смеси, детонационной шашки и запала. Корпус для увеличения числа осколков кроме двух наружных полусфер имеют две внутренние. Все четыре полусферы изготовлены из стали, нижняя наружная имеет наружную насечку, остальные - внутреннюю. В верхней части корпуса манжетой завальцован стакан для запала, при хранении прикрываемый пластмассовой пробкой. Под стаканом в углублении внутри взрывчатой смеси помещена детонационная шашка. Запал собран в пластмассовом корпусе, состоит из накольно-предохранительного механизма, датчика цели, дистанционного устройства, механизма дальнего взведения и детонирующего узла.

Накольно-предохранительный механизм обеспечивает безопасность в обращении с гранатой. После того, как выдернута чека гранаты, срабатывает механизм дальнего взведения, который взводит запал через 1-1,8 секунды после броска. Датчик цели обеспечивает мгновенное срабатывание запала при ударе о преграду. Дистанционное устройство обеспечивает замедление подрыва после броска на 3,2-4,2 секунды и дублирует датчик цели, если граната попадает в грязь, снег, падает строго «на бок».

Детонирующий узел закреплен в стакане и состоит из капсюля-детонатора и втулки. Сравнительно сложная конструкция запала обеспечивает сочетание безопасности обращения (6 ступеней предохранения) с гарантированным его срабатыванием. Температурный диапазон работы гранаты от -50 до +50 градусов С. Граната РГО носятся в стандартной гранатной сумке по две или в карманах снаряжения.

\begin{table}[h!]
	\caption{Боевые свойства оборонительной гранаты РГО}
	\label{tabular:RGO}
	\begin{center}
		\begin{tabular}{|l|l|}
			\hline
			\rowcolor[gray]{.9} Характеристики &  Граната РГО\\
			Масса гранаты, г.	& 530\\
			Масса боевого заряда, г.	& 92\\
			Дальность броска, м.	& 20-40\\
			Количество осколков,шт.  & 670-700\\
			Средняя масса осколков, г   & 0,46 \\
			Начальная скорость полета осколков, м &1200\\
			Площадь разлета осколков, м2      213-286 \\
			Время горения запала, сек & 3,2-4,2\\
			Радиус убойного действия осколков, м & 16,5\\
			\hline
		\end{tabular}
	\end{center}
\end{table}



\section{Назначение, боевые свойства, устройство гранат РГН.}
\textbf{Назначение, боевые свойства и общее устройство ручной осколочной гранаты РГН}

Ручная осколочная граната \index{РГН} предназначенная для поражения живой силы противника в наступлении и в обороне.

Ручная осколочная граната РГН (наступательная) разработана на предприятии «Базальт» в конце 1970-х годов. Существенное отличие этой гранаты от аналогичных образцов заключается в оснащении ее датчиком цели и срабатывании ее при ударе о любую преграду.
\begin{figure}[h!]
\centering
\includegraphics[width=0.2\linewidth, height=0.2\textheight]{img/RGN}
\caption[Ручная наступательная гранната РГО]{}
\label{fig:RGN}
\end{figure}

Граната состоит из корпуса, заряда взрывчатой смеси, детонационной шашки и запала. Корпус РГН образован двумя полусферами из алюминиевого сплава с внутренней насечкой. В верхней части корпуса манжетой завальцован стакан для запала, при хранении прикрываемый пластмассовой пробкой. Под стаканом в углублении внутри взрывчатой смеси помещена детонационная шашка. Запал собран в пластмассовом корпусе. Он состоит из накольно-предохранительного механизма, датчика цели, дистанционного устройства, механизма дальнего взведения и детонирующего узла.

Накольно-предохранительный механизм обеспечивает безопасность в обращении с гранатой. После того, как выдернута чека гранаты, срабатывает механизм дальнего взведения, который взводит запал через 1-1,8 секунды после броска. Датчик цели обеспечивает мгновенное срабатывание запала при ударе о преграду. Дистанционное устройство обеспечивает замедление подрыва после броска на 3,2-4,2 секунды и дублирует датчик цели если граната попадает в грязь, снег, падает строго «на бок».

Детонирующий узел закреплен в стакане и состоит из капсюля-детонатора и втулки. Сравнительно сложная конструкция запала обеспечивает сочетание безопасности обращения (6 ступеней предохранения) с гарантированным его срабатыванием. Температурный диапазон работы гранаты от -50 до +50 градусов С. Граната РГН носятся в стандартной гранатной сумке по две или в карманах снаряжения.
\begin{table}[h!]
	\caption{Боевые свойства наступательной  гранаты РГН}
	\label{tabular:RGN}
	\begin{center}
		\begin{tabular}{|l|l|}
			\hline
			\rowcolor[gray]{.9} Характеристики &  Граната РГН\\
			Масса гранаты, г.	& 310\\
			Масса боевого заряда, г.	& 114\\
			Дальность броска, м.	& 25-45\\
			Количество осколков,шт.  & 220-300\\
			Средняя масса осколков, г   & 0,42 \\
			Начальная скорость полета осколков, м &700\\
			Площадь разлета осколков, м2      &95-96 \\
			Время горения запала, сек & 3,2-4,2\\
			Радиус убойного действия осколков, м & 8,7\\
			\hline
		\end{tabular}
	\end{center}
\end{table}



ТО что я забыл
\section{Цель ведения разведки противника}
\textbf{Цель разведки} - добыть сведения о противнике и местности, необходимые командиру для принятия наиболее обоснованного решения и успешного ведения боевых действий. разведка должна установить: силы, состав и характер действий противника; своевременно и точно выявить место расположения средств применения ядерного оружия и подготовку к его применению, наличие и места расположения танков, артиллерии, противотанковых и других огневых средств, инженерных сооружений и заграждений, зоны заражения и районы разрушений, затоплений и завалов и другие объекты противника.

\section{Требования предъявляемые к разведке}
Основными требованиями, предъявляемыми к разведке, являются:\footnote{Можно было и расписать каждый пунк, но думаю что они все понятны.} 
\begin{itemize} 
\item целеустремленность
\item непрерывность,
\item активность
\item своевременность и оперативность
\item скрытность 
\item достоверность и точность определения координат разведываемых объектов (целей).
\item Особенно высокую точность должны иметь сведения о местонахождении средств ядерного и химического нападения противника.
\end{itemize}
\section{\textcolor{red}{!!!Разведывательные органы выделяемые от подразделения} }
\section{\textcolor{red}{!!!Наблюдательный пост и его оборудование}}
\section{\textcolor{red}{!!!Способы ведения разведки}}
\section{\textcolor{red}{!!!Различия между поиском и налетом}}

\section{При постановке задач наблюдательному пункту указывается}
При постановке задачи наблюдательному посту (НП) указываются:
\begin{itemize}
\item ориентиры (кодированные наименования местных предметов1);
\item сведения о противнике;
\item сведения о своих подразделениях (расположение передовых подразделений своих войск);
\item место наблюдательного поста (наблюдателя);
\item сектор (полоса, район, объект) наблюдения, что установить, за чем наблюдать, на что обращать особое внимание;
\item направления вероятного подлета самолетов (вертолетов) противника;
\item порядок доклада результатов наблюдения и сигналы оповещения.
\end{itemize}
\section{Схема ориентиров, зоны наблюдения-}

\section{\textcolor{red}{!!!Обязанности старшего наблюдательного поста}}
\newpage 

\section{Введение журнала наблюдателя}

\begin{table}[h!]
	\caption{Журнал наблюдения, пример}
	\begin{center}
		\begin{tabular}{|c| p{8cm} |c|}
			\hline 
			Время наблюдения & Где и что замечено & Кому у кода доложено\\
			\hline
			6.40 & Ор. 2, в право 20, ближе 200, у куста солдаты производили земляные работы & Майору Степанову в 6.45\\
			7.00 & Ор.1 дальше 300, в окопе наблюдатель противника & Ему же в 7.05 \\
			\hline
		\end{tabular}
	\end{center}
\end{table}

\section{Основные формы действий НВФ.}
Основу боевых действий против Федеральных войск РФ составляют методы и способы вооруженной борьбы с элементами партизанской войны. 

\textbf{Главными принципами ведения боевых действий НВФ являются: }
\begin{itemize}
	\item уход от прямых столкновений с превосходящими силами Федеральных войск 
	\item отказ от тактики позиционной войны, от удержания, занимаемые районов в течение длительного периода времени 
	\item широкое использование и сочетание с идеологической обработки личного состава местного МВД и населения 
	\item главный метод ведения боевых действий - внезапное нападение, так называемая "тактика Шамиля" (налет - отход). 
\end{itemize}
\textbf{Основными формами действий НВФ являются: }
\begin{itemize}
	\item засады 
	\item обстрелы 
	\item налеты 
	\item захват заложников 
	\item минирование маршрутов движения Федеральных войск и вблизи базовых центров Федеральных войск (ФВ), производят минирование дорог на подступах к ним, совершают диверсии и провокации. Для устройства засад, блокирования дорог и путей подвоза материальных средств 
\end{itemize}
\textbf{Тактика действий (НВФ) зависит от различных факторов: }
\begin{itemize}
	\item времени года 
	\item характер местности 
	\item этнического состава населения 
	\item количество и качество получаемого оружия, целей операции и других 
\end{itemize}
Характерными особенностями действий боевиков являются: 
\begin{itemize}
	\item проведение дисперсионно-террористических действий в населенных пунктах, на важных промышленных объектах и лечебных учреждениях 
	\item широкое применение мин, фугасов, в том числе управляемых, на дорогах и объектов 
	\item широкое применение гранатометов, как против бронеобъектов, так и против живой силы 
\end{itemize}

Боевиками постоянно осуществляется обстрелы позиций Федеральных войск, с целью спровоцировать их на ответные действия, причем провокации проводятся вблизи населенных пунктов, где проживают мирные жители. НВФ ведут постоянную разведку, руководство НВФ может использовать местное население (стариков, женщин, детей), прикрываясь ими в качестве живого щита. Большое внимание руководство НВФ уделяет иностранным военным специалистам и наемникам. На них возлагается большинство задач по проведению террористических актов и диверсии против Федеральных войск, обучение боевиков способам и методам партизанской войны, карательные функции. 



\section{Этапы подготовки и проведения террористических актов НВФ.}
\textbf{ЗАСАДА}

В состав засадной группы, как правило, входят: 
\begin{itemize}
	\item наблюдатели 
	\item огневая подгруппа, предназначенная для поражения живой силы и техники 
	\item подгруппа предупреждения - для ограничения маневра противника 
	\item резервная подгруппа - для усиления огневой подгруппы и прикрытия отхода 
\end{itemize}
Перед устройством засады боевики осуществляют тщательную разведку района. Сбор информации осуществляется специально подготовленными разведчиками, а также через родственников боевиков, проживающих вблизи мест расположения частей и подразделений Федеральных войск. Для ведения разведки наблюдением могут привлекаться и подростки из числа местного населения. На основании полученной информации заблаговременно подготавливают огневые позиции, определяют пути отхода и место последующего сбора группы. Огневая подгруппа, включает в себя основные силы НВФ (2 - 3 гранатомета, несколько снайперов, в отдельных случаях минометы), обычно располагаются на удалении, позволяющий вести эффективный огонь. Боевики этой подгруппы размещаются вдоль дороги, дефиле, узкой на расстоянии 50 - 100 метров и в 10 - 15 метров друг от друга. Подгруппа предупреждения занимает позиции на направлении вероятного отхода или маневра. При входе колонны в зону поражения снайпера первыми открывают огонь по водителям и старшим головных автомашин. Другие боевики одновременно обстреливают личный состав и ведут огнь по бронированным целям из РПГ и пулеметов. Основная цель - нарушить управление колонной, создать панику и условия для ее раздробления, уничтожения по частям, или захвата. Проведение засад шаблона не имеет, они возможны, как утром, так и во второй половине дня, перед наступлением темноты с целью исключит возможность применения авиации со стороны Федеральных войск. Иногда боевики применяют специальную отвлекающую группу, которая располагается впереди основных сил засады. Эта группа открывает внезапный огонь по колонне и стремится сковать боем подразделения охраны. В это время, прошедшая вперед уже без охранения колонна, попадает под огонь главных сил засады НВФ и несет большие потери. При организации засады в населенных пунктах боевики скрываются за заборами, в домах и строениях, стараются не обнаруживать себя и беспрепятственно пропускают разведку и органы охранения. С приближением главных сил открывают огонь из бойниц, окон и дверей домов, сосредотачивая его, в основном, по автомобилям, перевозящим личный состав. В момент открытия огня, как правило, производится прицельный залп из стрелкового оружия, минометов и гранатометов, затем ведется интенсивный обстрел отдельных целей. В зеленых зонах засады устанавливаются на путях вероятного движения подразделений путем внезапного обстрела, как с фронта, так и с флангов. При этом засады могут, осуществляется последовательно с нескольких рубежей по мере продвижения техники и личного состава. Встретив сопротивление, боевики немедленно покидают место засады, стремясь избежать боя. Наибольшую опасность для них представляет расчленение группы и выход в их тыл. При хорошо организованной разведке и охране колонн силами сопровождения, а также прикрытия с воздуха, боевики, обычно, не рискуют нападать на колонны. Обстрелы блокпостов, контрольно - пропускных пунктов и комендатур ВОС. Места обстрелов, в зависимости от конкретных условий, выбирается по возможности с близкого расстояния. Перед обстрелом осуществляется тщательная разведка объекта. Отмечены случаи обстрелов реактивными снарядами, из минометов и стрелкового оружия с подвижных средств (автомобилей), что позволяет группе приблизится к объекту, произвести обстрел и быстро выйти за пределы зоны эффективного ответного огня. 

\textbf{НАЛЕТ НА ОБЪЕКТ }

Налету предшествует тщательная разведка объекта (система охраны, ограждение, пути отхода, возможность усиления и т. д.). Организуя налет на объект, боевики составляют план действий, включая элементы скрытного приближения к месту нападения, обеспечения безопасности в ходе налета, быстрого отхода с применением маневра. Пред налетом обычно проводятся тренировки в условиях, максимально приближенных к реальным. При налете НВФ могут действовать укрепленными группами и отрядами численностью в несколько десятков человек. При этом их боевой порядок включать отвлекающую группу, которая выдвигается к объекту атаки в полный рост, и ударную группу, действующую скрытно, преимущественно ползком и выполняющую главную задачу. Обычно эта группа состоит из двух подгрупп. Оптимальный состав ударной группы 10 - 15 чел. 
Она включает в себя подгруппы: 
\begin{itemize}
	\item подавления 
	\item прикрытия 
	\item налета 
	\item инженерная 
\end{itemize}
Их задачами являются: 
\begin{enumerate}
	\item обезвреживание часовых .
	\item перекрытие путей отхода и маневра 
	\item обеспечение проходов в заграждениях 
	\item захват (удержание объекта) 
	\item прикрытие отхода после выполнения задания 
\end{enumerate}
НАПАДЕНИЕ И ЗАХВАТ НАСЕЛЕННЫХ ПУНКТОВ 

Нападение на населенные пункты предшествует тщательная подготовка. Прежде всего, боевики организуют сбор данных о расположении постов охраны в тех или иных населенных пунктов, где они имеют осведомителей из числа местных жителей, сотрудников местных самоуправления и милиции. Затем осуществляется подготовка, в ходе которой в данном населенном пункте в садах и дворах местных жителей скрытно оборудуется тайники, в которых заблаговременно создаются запасы оружия и боеприпасов. Часть боевиков под видом мирных жителей, родственников или знакомых по 1 - 2 человека проникают в селение и ждут установленного сигнала. Основная часть банды во главе с руководителем на автотранспорте, как правило, рано утром, прибывает в населенный пункт. К ним примыкают та часть боевиков, которая заранее обосновалась. Боевики уничтожают или блокируют отделы МВД, посты и расположения Федеральных войск, занимают административные здания и захватываю заложников. В случаи захвата населенного пункта, действия боевиков носят характер, свойственный басмаческим бандам: осуществляются расстрелы представителей местной власти, грабеж населения (в основном русскоязычного) и поджоги. В случаи захвата того или иного населенного пункта и намерения закрепится в нем, боевики осуществляют мероприятия по подготовке к ведению оборонительных действий. В этих целей они, привлекая местных жителей, оборудуют огневые позиции, ходы сообщения, разрабатывают систему огня, минируют подходы к населенному пункту. На крышах домов выставляют наблюдателей, в глубине помещений складывают боеприпасы и ВВ. Для пулеметов, минометов и РПГ подготавливаются несколько огневых позиций. При приближении войск боевики открывают сосредоточенный огонь, а затем уходят в глубь населенного пункта, на новый рубеж обороны. При налете авиации и обстреле артиллерией они укрываются в специально сооруженных укрытий. По окончании обстрела занимают прежние сооружения. Действия боевиков в населенных пунктах становится более организованными, и приобретают обдуманный, изощренный характер. Так, при ведении боевых действий в Гудермесе, НВФ заняли позиции по рубежам на значительную глубину и создали отдельные узлы сопротивления, в которых установили тяжелое оружие. Отмечены случаи, когда отдельные районы города заблаговременно оборудовались в инженерном отношении. Вдоль улиц отрывались траншеи, ходы сообщения, оборудовались блиндажами, долговременные огневые точки с несколькими амбразурами. Подступы к огневым точкам, улицы и переулки минировались противотанковыми и противопехотными минами. В домах, не занятыми боевиками, устанавливались мины - ловушки (на дверях, окнах, предметах домашнего оприхода). Нападения и захват населенных пунктов преследуют цель: создать напряженность среди жителей, подорвать их веру в способность власти вести эффективную борьбу против боевиков, вынудить их пополнить ряды боевиков или уходить в соседние районы и республики, для формирования (усиления) там чеченской диаспоры. 

БОРЬБА С АВИАЦИЕЙ 

НВФ уделяет большое внимание вопросам борьбы с самолетами и вертолетами во время нанесения ими огневых ударов в ходе авиационной поддержки действий войск. Основными средствами борьбы являются крупнокалиберные пулеметы, ПЗРК, ЗУ, ЗСУ. В горных районах практикуется также установка зенитных средств под карнизами с соответствующим сектором стрельбы. В борьбе с вертолетами огонь зенитных средств обычно сочетается с залповым заградительным огнем из стрелкового оружия. Кроме того, для стрельбы по воздушным целям боевики используют автоматическое оружие обычного калибра с импровизированных "станков" (веток деревьев и других местных предметов). Рассредоточение зенитных средств обеспечивает боевикам ведение огня одновременно с различных направлений, что затрудняет их попадание. Боевики тщательно изучают уязвимые места самолетов и вертолетов, маршруты полетов авиации, ведут за ними наблюдение. На вооружении НВФ имеются также современные средства для поражения воздушных целей на малых (300 - 500 м) и средних (2500 - 3000 м) высотах. Руководство НВФ настойчиво ищет возможность увеличения поставок. 

МИНИРОВАНИЕ 

Минирование дорог, разрушение мостов и коммуникационных линий связи, также являются одним из способов ведения НВФ вооруженной борьбы. Установка мин осуществляется специальными группами (4 - 5 чел.), которые в качестве помощников привлекают местных жителей и подростков после их продолжительной подготовки. Наиболее часто мины закладываются на магистральных (с твердым покрытием) и полевых дорогах, используемых войсками. Как правило, для поражения боевой техники и транспортных средств на проезжей части дороги применяются противотанковые противопехотные мины, иногда ящики с ВВ. в ряде случаев боевики используют мощные фугасы: артиллерийские снаряды и авиабомбы, 2 - 4 противотанковые мины, уложенные одна на другую, либо одновременно противотанковые и противопехотные. Как правило, используют 3 способа минирования маршрутов: 
1.	упорядоченный 
2.	неупорядоченный 
3.	смешанный 
При непорядочном способе расстояние между минами составляет 1.5 - 2 м. При минировании на снегу мины перекрашивают в белый цвет. В последнее время боевики стали применять на дорогах минирование "цепочкой" - установка 30 - 40 мин на участке длинной 200 - 300 м, что приводит к увеличению количества подрывание транспортных средств и личного состава. Новым элементом является применение мини-фугасов, начиненных бензином, керосином или дизтопливом. В этом случае при взрыве происходит разбрызгивания горящего вещества и воспламенения не только подорвавшего объекта, но и рядом находящихся. Отмечены случаи, когда боевики закапывали открытые пузырьки с бензином или керосином рядом с минами, с целью их затруднить их обнаружения розыскными собаками. НВФ применяют также управляемые мины и фугасы, предназначенные для поражения конкретных, заранее избранных целей, например, машина управления, руководителей местных и федеральных органов власти, командования Федеральных войск. Особое место в действиях боевиков отводится проведение диверсионных и террористических актов. Их главными объектами являются линии связи и электропередачи, государственные и культурно-просветительные учреждения, хозяйственные предприятия, сотрудники федеральных и местных органов власти, представители интеллигенции. Применяя в широких масштабах террор и диверсии, боевики стремятся создать в республике обстановку страха и неуязвимости. Диверсии проводятся силами специально подготовленных групп (отрядов) различной численности. Обычно группа разделяется на части, каждая из которых выполняет свою задачу. Так, первая совершает нападение на охрану объекта, вторая (техническая) обеспечивает осуществление диверсии непосредственно на объекте, третья предназначается для вывода из строя линии связи и ведения боя с подкреплениями. Как правило, эта часть группы располагается в укрытиях, обеспечивающих надежную маскировку и имеет на вооружении легкие пулеметы, автоматы и гранатометы. Диверсии совершаются обычно через 1 - 2 часа после наступления темноты. Наиболее характерными видами диверсий являются подрыв военной техники, вывод из строя трубопроводов, разрушение зданий органов власти, аэровокзалов, больниц, гостиниц и других объектов жизнеобеспечения и социального назначения. Для уничтожения военной техники, которая на всю ночь возвращается в места дислокации подразделений, боевики минируют места стоянок и чаще подступы к ним. Для разрушения зданий боевики используют мины и фугасы, к установке которых привлекается специально подготовленный персонал.


\section{Требования, предъявляемые месту для окопа.}
При расположении сооружений на местности необходимо учитывать ее защитные и маскирующие свойства. Например, место для окопа необходимо выбирать так, чтобы иметь хороший обзор и обстрел в заданном секторе и не быть заметным для противника. В то же время расположение окопов зависит от поставленной подразделению боевой задачи и условий местности.

Окопы могут располагаться на передних и обратных скатах высот. Наиболее удобными местами для их расположения являются передние скаты. Расположение окопов на топографическом гребне затрудняет наблюдение и обстрел ближних подступов из-за большого количества мертвых пространств. При расположении окопов на обратном скате они должны находиться не ближе 200 м к топографическому гребню.



\section{Порядок введения наблюдения.}




\section{Оборудование взводного опорного пункта}
   Инженерное оборудование опорного пункта осуществляется в соответствии с указаниями командира роты по инженерному обеспечению, а также решением командира взвода на бой.
   Инженерное оборудование опорного пункта включает:
   \begin{itemize}
   	\item  установку проволочных и других заграждений;
   	\item  расчистку полос обзора и обстрела;
   	\item оборудование позиций отделений и сплошной траншеи во взводном опорном пункте;
   	\item оборудование окопов для боевых машин пехоты (бронетранспортеров), танков, противотанковых управляемых ракет и других огненных средств;
   	\item возведение сооружения для командно-наблюдательного пункта взвода;
   	\item устройство перекрытых щелей, блиндажей;
   	\item отрывку хода сообщения в глубину.
   \end{itemize}
   

   На всём фронте опорного пункта мотострелкового взвода отрывается сплошная траншея, которая соединяет одиночные (парные) окопы для мотострелков, окопы для БМП, танков, установок ПТУР, других огневых средств, располагающихся в его пределах, укрытия для личного состава.
   
   Если одно из отделений взвода располагается в глубине опорного пункта (во второй линии), то на него отрывается окоп по фронту до 100 м, который соединяется ходом сообщений с траншеей. От опорного пункта в глубину обороны также отрывается ход сообщений, который подготавливается для ведения огня. По периметру опорного пункта в целях обеспечения круговой обороны могут отрываться участки траншей (4-5), каждый протяженностью 15-20 м, с ячейками для стрельбы. Траншеи, окопы и ходы сообщения должны быть подготовлены для ведения флангового и перекрестного огня по атакующему противнику, обеспечивать возможность проведения скрытного маневра и рассредоточения огневых средств, а также введение противника, в заблуждение относительно истинного построения взводного опорного пункта.
   
   На оборудование опорного пункта взвода первого эшелона роты (с расположением позиций отделений в две линии) требуется 1100 чел.-час. и 7 маш.-час. танка с бульдозерным оборудованием. При оборудовании опорного пункта второго эшелона роты (с расположением позиций отделений в одну линию) потребуется 1250 чел.-час. и 7 маш.-час. танка с бульдозерным оборудованием. Инженерное оборудование позиции отделения включает: сооружение окопа на отделение протяженностью до 100 м, состоящего из участка траншеи с ячейками (бойницами) для стрелков, площадок (ячеек) для пулемета и гранатомета; отрывку окопа на БМП (БТР); сооружение блиндажа (перекрытой щели); устройство ниши для боеприпасов и продовольствия, тупиков и уширений для удобства передвижения; отрывку хода сообщения в глубину опорного пункта. На оборудование позиций пехотной лопатой требуется 300-400 чел.-час., при использовании саперной лопаты 150-200 чел.-час.
   Проволочные и другие заграждения перед опорным пунктом взвода и на флангах устраиваются в целях нанесения противнику потерь в живой силе и технике, задержки его продвижения и сковывания маневра.
   
   Заграждения устанавливаются как при подготовке обороны, так и в ходе ее ведения силами взвода и инженерных подразделений. Личный состав взвода в период подготовки обороны обычно устраивает невзрывные заграждения: рвы, эскарпы, контрэскарпы, надолбы, заграждения из колючей и гладкой проволоки (малозаметные проволочные сети, спирали, рогатки, ежи и др.). Иногда (при недостатке инженерных подразделений) взвод может привлекаться для устройства и минно-взрывных заграждений строевым расчётом или по минному шнуру.
   
   В ходе ведения боя для борьбы с пехотой противника перед передним краем, при вклинении ее на фланги взвода или в промежутки между позициями Отделений может применяться переносным комплект Минирования (ПКМ), способный па любом угрожаемом участке дистанционно создать минное поле площадью 200-400 м2 на дальности до 100 м от позиции.
   
   В опорном пункте танкового взвода не отрываются сплошные траншеи, а оборудуется лишь окоп на позиции приданного мотострелкового отделения. Для танков оборудуются окопы на основных и запасных, (временных) огневых позициях. Одна из основных огневых позиций оборудуется двумя окопами и укрытием, соединенными между собой траншеей. Окопы подготавливаются для кругового обстрела, участки траншеи оборудуются площадками для ведения огня в сторону флангов и в тыл опорного пункта.
   В целях создания наилучших условий для наблюдения и ведения огня перед фронтом обороны взвода и на флангах местность расчищается от кустарника и высокой травы.

\section{Hello}

\begin{proof}
	Вот мои пруфы  
	
\end{proof}
\begin{proof}
	Вот мои пруфы  
	
\end{proof}
\begin{proof}
	Вот мои пруфы  
	
\end{proof}
\begin{proof}
	Вот мои пруфы  
	
\end{proof}
\begin{proof}
	Вот мои пруфы  
	
\end{proof}
\begin{proof}
	Вот мои пруфы  
	
\end{proof}
\chapter{Практическая часть}
\begin{enumerate}
	\item Неполная разборка и сборка после разборки автомата Калашникова АКС-74 (нормативы №13,14).
	\item Снаряжение патронами магазина автомата КАлашникова, СВД (норматив №15).
	\item Надевание противогаза или респиратора (норматив №1) 
	\item Надевание общевойскового защитного комплекта, костюма защитного плёночного и противогаза (норматив №4(а,б))
\end{enumerate}

%\printindex
\end{document}
